\chapter{Introduction}
\label{ch:intro}

%    Move 1 establish your territory (say what the topic is about)
%    Move 2 establish a niche (show why there needs to be further research on your topic)
%    Move 3 introduce the current research (make hypotheses; state the research questions)


%    state the general topic and give some background
%    provide a review of the literature related to the topic
%    define the terms and scope of the topic
%    outline the current situation
%    evaluate the current situation (advantages/ disadvantages) and identify the gap
%    identify the importance of the proposed research
%    state the research problem/ questions
%    state the research aims and/or research objectives
%    state the hypotheses
%    outline the order of information in the thesis
%    outline the methodology

% https://www.scribbr.com/theses-examples/examples-dissertation-phd-theses/

% Soft matters ties to optical microscopy.
% Why advances in optical microscopy are necessary/fruitful.
% The advance of quantitative microscopy and HVM
% Outline of thesis in a narrative format.

% fluorescent labeling, super-resolution microscopy, digital microscopy
% provide increased localization.
% While ray optics paired with expensive optics leads researchers to
% interpret each microscopic image as a 1-to-1 depiction of life in the
% sample, the reality is much more complicated. Diffraction, real
% optics, focus.. images are complicated.
% Reseachers who want access to the deluge of quantitative
% information able via microscopy have 



\section{The pursuit of quantitative microscopy}

Microscopy has revolutionized the physical sciences by producing
detailed images of features that are invisible to the naked eye.
Optics manufacturing has 
Modern bright-field microscopes image samples up to the
Abbe diffraction limit \cite{abbe1873}
and several super-resolution techniques now exist for
circumventing this historic limit.
Many methods for extracting
quantitative information from microscopic images rely
on interpreting regions of contrast as features and
then 

Here we describe holographic particle characterization,
a quantitative microscopy technique, for extracting
the three-dimensional position, size, and refractive index of
colloidal scatterers. By 

\section{Organization}

% FIXME: Make Chapter part of the href.
Chapter \ref{ch:hvm} provides an extensive theoretical and practical
introduction to holographic particle characterization. The primary
experimental setup for all subsequent experiments is introduced.

In \autoref{ch:debye} we model the optical train to establish the
physical limits of HPC set by the imaging apparatus. The scattered
field and incident field are evaluated at several planes along the
optical train. The refraction of rays through optical element
may impart a phase shift, a polarization rotation and a scaling
of complex magnitude. We determine that a scalar theory
approximation successful accounts for the images arriving in the
focal plane for the range of scatterers we are interested in.

Chapter \ref{ch:dimpled} presents a series of experimental and
simulated studies to determine the effective of Lorenz-Mie
microscopy in characterizing slight aspherical particles.
Specifically, we synthesize three types of colloidal particles:
spheres, spheres will small dimples and spheres with large dimples.
The work of this chapters validates the use of the Lorenz-Mie theory
for analyzing spheres which have small geometrical deviations from
ideal sphericity. Additionally, our work determines that the Lorenz-Mie
theory acts as an effective sphere model for particles with
pronounced deviations from a spherical geometry and provides
accurate information on the scatterer.

The first step in holographic particle characterization
requires detecting and localizing holographic features in experimental
images. Heuristic algorithms perform admirably for isolated, homogeneous
features; after tuning the necessary empirical parameters, they achieve
sub-pixel localization while incurring few false positives. However their
performance deteriorates when presented overlapping features or even
heterogenous holographic features (axial range, size and refractive index).

We present in \autoref{ch:cascade} two machine-learning algorithms for
detecting and localizing holographic features: a convolution
neural network (CNN) and a cascade classifier. Direct comparison with the
heuristic algorithm over a range of experimental and synthetic images
shows promising results. The CNN demonstrates robust feature detection
for overlapping holograms or even highly heterogeneous holographic features
in the same image. The cascade classifier's localization precision
offers is a ten times worse than the CNN, but is also ten times faster than
the CNN. In fact the cascade classifier is fast enough for real-time
applications such as high-speed targeting in holographic optical
trapping systems.

After detecting a holographic feature, an initial estimate of the
scatter's size, refractive index, and three-dimensional position
is necessary for the subsequent fit to the Lorenz-Mie theory.
For homogeneous samples of a single particle type, the fitting procedure
can be bootstrapped with outside information. For heterogeneous samples
with disparate sizes or refractive indices, the fitting procedure
will likely lock onto local minima and return erroneous results for
the size and refractive index. holographic features to the Lorenz-Mie theory requires
In \autoref{ch:svr} we demonstrate the use of a support vector machine for
estimating the size and refractive index of scatterers from snapshots
of holographic features. Our implementation has several steps. Our fitting
algorithm usually fits the intensity pattern from \SI{40000}{} pixels;
We reduce the dimensionality of this problem by analyzing the radial
profile of the image made by taking an average over angles. In addition
we produce a SVM for the three properties we need to bootstrap:
size, refractive index, and height above the focal plane. We demonstrate these
support vector machines can resolve heterogeneous samples with four greatly
differing particle properties.

In the previous chapters we have worked towards validating and extending
the domain of applicability of holographic particle characterization.
In \autoref{ch:synthesis} we close with a demonstration of its use for
analyzing the synthesis of \num{3}-methacryloxypropyltrimethoxysilane (TPM) colloidal
spheres. The synthesis is principally a two-step procedure: a emulsion of monodisperse,
micrometer sized TPM droplets is prepared and then polymerized.
A number of experimental parameters and conditions can however affect
the resulting size of the polymerized spheres. In this chapter we use
holographic particle characterization to analyze liquid droplets and
solid TPM spheres. By synthesizing particles under a number of
differing protocols, we isolate a number of experimental conditions
that affect particle size.
