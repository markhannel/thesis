\chapter{Holographic Video Microscopy}
\label{ch:hvm}

% Suggested figures.
% Select Historical contributions.
% Diagrams depicting the physical processes
% sequentially.j

\newcommand{\einc}{\vec{E}_{\text{inc}}}
\newcommand{\escat}{\vec{E}_{\text{s}}}
\newcommand{\eadd}{\vec{E}_{\text{add}}}
\section{The Advent of Holography}

% Review of literature and historical perspective.



\section{Holographic Video Microscopy}


Holographic Video Microscopy differs from
traditional microscopy techniques by preserving,
collecting, and making quantitative use of phase
information; this information is
%Traditional microscopy utilizes
%incoherent illumination and therefore all
%phase information is 
\subsection{Broad Overview}

% The importance of phase information retrieval.

%

% Sample preparation.
% Cleaning and dilution of sample necessary
% due to the depth of FOV.

\begin{eqnarray}
  I &=& \abs{\einc + \escat}^2\\
    &=& \abs{\einc}^2 + \einc\cdot\escat^* + \einc^*\cdot\escat + \abs{\escat}^2\\
    &=& \abs{\einc}^2 + 2\Re\left \{\einc\cdot\escat^*\right \} + \abs{\escat}^2
\end{eqnarray}

\begin{eqnarray*}
  I &=& \abs{\einc + \escat + \eadd}^2\\
    &=& \abs{\einc}^2 + 2\Re\left \{\einc\cdot\escat^*\right \} + \abs{\escat}^2
\end{eqnarray*}


\subsection{Illumination}



\subsection{Scattering}

  

\subsection{Collection and Refocusing}

\subsection{Digital Recording}

\section{The Inverse Problem}

\subsection{Feature Detection}

\subsection{Characterizing Holograms}
