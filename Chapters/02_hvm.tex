\chapter{Holographic Video Microscopy}
\label{ch:hvm}

% Suggested figures.
% Diagrams depicting the physical processes
% sequentially.
% The setup.

\newcommand{\einc}{\vec{E}_{\text{inc}}}
\newcommand{\escat}{\vec{E}_{\text{s}}}
\newcommand{\eadd}{\vec{E}_{\text{add}}}


\section{Overview}

%In this chapter, we will review the salient details 
%underlying the theory and experimental implementation
%of holographic video microscopy. Whenever possible,
%extraneous details that detract from our narrative
%will be relegated to the appendix or be given 
%primary source citations.


Holographic video microscopy illuminates a sample of scatterers with
coherent illumination in order to preserve phase information. 
The resulting scattering pattern is then
fit to an appropriate light scattering theory to measure physical properties
of each scatterer, primarily their size, refractive index, and
three-dimensional position.

In this chapter, we will acquaint the reader with our primary experimental
setup and in so doing outline the salient physical processes underlying
HVM. We will then describe the Lorenz-Mie theory of light
scattering and discuss the limited validity of common assumptions.
We will then discuss our implementation our image analysis before
concluding with a number of exciting applications of holographic video
microscopy.


\section{Experimental Setup}
\label{ch:hvm:sec:hvm}

A diagram of our custom-built in-line holographic microscope
is provided in Fig.~%\ref{fig:hvm_01}.
Our setup illuminates
the sample plane with a blue ($\SI{447}{nm}$ vacuum wavelength),
linearly polarized laser beam (Coherent Cube). The beam's
$\SI{25}{\mW}$ of power is spread over the $\SI{3}{\mm}$ beam
diameter, producing a peak irradiance of approximately
$\SI{0.88}{\mW / \mm^2}$. Before scattering through the
sample, the beam passes through a quarter-wave plate and a
polarizing beam splitter (ThorLabs CCM1-PBS251) to enable
beam attenuation and to optionally provide bright-field
illumination.

The coherent illumination and scattered light are collected by a
standard microscope objective (Nikon Plan Apo, $\num{100}$x,
numerical aperture $\num{1.45}$, oil immersion) and then focused
by a $\SI{200}{\mm}$ lens onto a greyscale digital camera
(NEC TI-324AII). The digital camera digitizes the resulting intensity
patterns to $8$-bits per pixel at $\SI{29.97}{frames / \sec}$ and relays the
resulting array of information to either a DVR (Pioneer 520HS) to record
on a DVD or directly to the hard disk of a personal computer.
Each pixel in the $\si{640 x 480}$ array of pixels has a width of
$\SI{13.5}{\um}$. After $100$x magnification, a pixel in the
image has an effective size of $\SI{0.135}{\um}$.

% Description of trapping?


  % BLURB connecting HVM to DHM
%Holographic video microscopy is one of several
%digital holographic microscopy (DHM) techniques.
%DHM differs from traditional bright-field microscopy
%by preserving, collecting, and making quantitative use of phase
%information. In most variants of DHM, each recorded
%wavefront can be numerically reconstructed, or rather
%{\it un}-propagated, all the way back to a scattering event
%to produce an image of the scatterer. By reconstructing
%several planes around the scatterer, each hologram
%can provide insight into the topography of each scatter
%in the field of view.

%Holographic video microscopy differs from DHM by analytically
%fitting the scatter's properties to the resulting


\subsection{Image formation}
\label{ch:hvm:sec:hvm:ssec:overview}

The scatterers responsible for the resulting holographic image
have their physical properties encoded in the image. To extract
this wealth of information, we will fit account 


\begin{equation}
  \label{eq:incidentfield}
  \einc(\vec{r}) = u_0(\vec{r}) \, e^{i \varphi_0(\vec{r})} \, e^{i k z} \, \hat{x},
\end{equation}

\begin{eqnarray}
  I &=& \abs{\einc + \escat}^2\\
    &=& \abs{\einc}^2 + \einc\cdot\escat^* + \einc^*\cdot\escat + \abs{\escat}^2\\
    &=& \abs{\einc}^2 + 2\Re\left \{\einc\cdot\escat^*\right \} + \abs{\escat}^2
\end{eqnarray}

\begin{eqnarray*}
  I &=& \abs{\einc + \escat + \eadd}^2\\
    &=& \abs{\einc}^2 + 2\Re\left \{\einc\cdot\escat^*\right \} + \abs{\escat}^2
\end{eqnarray*}


\subsection{Scattering}
\label{ch:hvm:sec:hvm:ssec:scattering}
  

\subsection{Digital Recording}
\label{ch:hvm:sec:hvm:ssec:digitalrec}

% What are digital images. How are digital images recorded?
% Why are you telling them such things?
% How is it that digital images do not record intensity?

Images in science are ubiquitous; by sampling the intensity over an exposure
period, a single image encodes a bevy of spatial information that is ripe
for qualitative and quantitative analysis. Historically, researchers would
extract this information by hand. Jean-Perrin used a projector, pen and a paper to originally
measure the diffusion of a polymer bead. Crystallographers such as %FIXME: Add reference.
traced photographs of crystalline diffraction patterns by hand. Presently researchers utilize
digital cameras to record and digitize images as arrays of values. 

We utilize a digital camera to record holograms. To fit our experimental holograms to
theory and to simulate our digital recordings it is necessary to have a basic
understanding digital images. In this section we will 

\subsubsection{Digitizing Images}
\label{ch:hvm:sec:hvm:ssec:digitalrec:sssec:digitizing}

Modern digital cameras employ an array of photon detectors to measure the
average intensity over the sensor surface. Each photon detector, referred to as a pixel,
utilizes the photoelectric effect to convert photons into excited electrons.
The excited electrons at each pixel are counted, sometimes to single precision, and
then digitized into $8$-, $12$-, or even $16$-bit integers. Because the number of
excited electrons is proportional to the number of photons, and the the number of
photons is in turn proportional to the intensity of the of the image at the
sensor surface, the array of electron counts serves as a proxy for the intensity
of the image.

With millions of pixels and the capacity to count tens of thousands of electrons at each pixel,
digital cameras are an engineering marvel. For our purposes, we will review a few of the physical
details of a single pixel to explain important imaging effects such as dark counts, saturation,
and the digitization procedure.

Pixels are designed to accurately count the number of electrons that are
excited by incident photons within a given exposure period. A number of physical constraints
limit the accuracy of each pixels.

During an exposure period, $N_p$ photons of a particular wavelength arrive at a pixel
surface. Some fraction of the incident photons are converted to excited electrons
with a wavelength dependent probability known as the quantum efficiency. To be properly
counted, these excited electrons must survive until the counting procedure has
accounted for their presence. To this end, each pixel is doped to increase the lifetime
of excited electrons. In addition, the excited electrons must remain in the bulk so that
they are not grounded; for this purpose, a biased field is applied. % FIXME: What about mirror charges?

During the exposure period, a number of electrons can be thermally excited (as opposed to
photonically excited) and will be included in the count. These erroneously
counted electrons are referred to as a dark count. Measuring the average dark count of a
camera is as easy as recording the average image while an opaque object is blocking the
camera's sensor. Note that the dark count will increase with the length of the exposure
period.

% The number of electrons that can be negative.. scientific cameras have a non-zero
% floor to maintain gaussian-errors.
% Saturation occurs because the relation number of excited electrons per
% number of incident photons becomes non-linear. The largest number of reported
% electrons is the highest level

% A number of approximations come to mind with the LM theory.
% Approximation of radial component
% Approximation of functional form (Hankel function)
% Approximation of polarization


\section{Implementing Holographic Video Microscopy}

In this section we will describe the details of our microscope and
the experimental methods that were implemented in the ensuing chapters.



\subsection{Instrumentation}

We
% Illumination
% Objective
% Tube lens
% Camera


\subsection{Sample Preparation}

% Colloidal Synthesis.
% Flow Cells.
% Sealed Samples.



\subsection{Image Analysis}

% Normalization.
% Detecting and Localizing Features
% Fitting with Levinberg-Marquardt
% Mention Dimiduk's Bayesian analysis.

\subsection{Feature Detection}

After appropriately normalizing an image, it is necessary to detect if any holographic
features are in the field of view. 


\section{Applications of HVM}

% Broad uses exemplified in this thesis.
% Uses elsewhere
