\chapter{Holographic Particle Characterization}
\label{ch:HPC}

\section{Overview}
\label{sec:overview}

In-line holographic video microscopy \cite{sheng06,lee07,lee07a} 
is a technique that creates holograms of individual objects in 
the microscope's field of view.
A holographic microscope illuminates
its sample with a collimated laser beam rather than
a conventional incoherent light source.
Light scattered by an object in the sample therefore
interferes with the remainder of the beam in the focal
plane of the microscope.  The microscope
magnifies this interference pattern and projects it onto the
face of a video camera, which records the spatially varying
intensity pattern, $I(\vec{r})$.
Each image in the resulting video stream is a holographic
snapshot of the scatterers passing through the laser beam. 
The scattering pattern due to an individual object such as a 
colloidal sphere is identified
as a center of rotational symmetry
and can be fitted pixel-by-pixel with predictions 
\cite{lee07a,cheong10a,krishnatreya14} based on the
Lorenz-Mie theory of light scattering 
\cite{bohren83,mishchenko02,gouesbet11}
to measure the object's radius, $a_p$, refractive index, $n_p$,
and three-dimensional position, $\vec{r}_p$.

\section{Holographic video microscope}
\label{sec:microscope}

\begin{figure}[h!]
  \centering
  \includegraphics[width=1.0\columnwidth]{setup}
  \caption[Schematic representation of a holographic video microscope]
  {Schematic representation of a holographic video microscope.}
  \label{fig:setup}
\end{figure}

Our custom-built holographic microscope \cite{krishnatreya14} 
is shown schematically in Fig.~\ref{fig:setup}. It uses a standard microscope objective lens (Nikon Plan Apo, $100\times$, numerical aperture 1.45, oil immersion) in combination with a tube lens to attain a system magnification of
\SI{135}{\nm\per pixel} on the face of an 8-bit monochrome
video camera (NEC TI-324AII), which records its intensity
at \SI{29.97}{frames\per\second}.
We illuminate the sample with the collimated laser beam
from a fiber-coupled diode laser operating at a vacuum wavelength 
of \SI{447 \pm 1}{\nm} (Coherent Cube), which delivers
\SI{25}{\milli\watt} of light to the specimen. 
This light spreads over the \SI{3}{\mm} diameter of the beam,
yielding a peak irradiance of \SI{3.5}{\milli\watt\per\square\mm}.

\section{Holographic characterization procedure}
A typical hologram subtends a $200 \times 200$ pixel array.
Each pixel has a relative noise figure of 0.009, as determined
by applying the median-absolute-deviation (MAD) metric to typical experimental images.
The 100:1 signal to noise ratio therefore allows each hologram to contain l lot more information than a conventional in-plane image.
A conventional image by contrast subtends a $10 \times 10$ pixel array so has less than one percent as much information. The additional information encoded in a hologram greatly improves the quantity and quality of data that can be extracted from holographic microscopy relative to conventional images.
For micrometer-scale colloidal spheres, the holographic characterization typically yields an individual particle's radius 
with nanometer precision \cite{cheong09,shpaisman12},
its refractive index to within a part per thousand 
\cite{cheong09,moyses13}, and
can track its position to within a nanometer in the plane
and to within 3 nanometers along the optical axis
\cite{lee07a,cheong09,cheong10a,moyses13,krishnatreya14}. Fits are performed with a pixel-by-pixel nonlinear least-squares regression algorithm to minimize the $\chi^2$ between the measured and theoretical light intensity pattern. This technique has been shown to work reliably for colloidal spheres ranging in radius from \SI{300}{\nm} to \SI{10}{\mu m} \cite{lee07a}.
Each fit can be performed in a few tens of milliseconds
using automated feature detection \cite{krishnatreya14a}
and image recognition algorithms \cite{yevick14}.

To characterize a population of micro-particles, we flow the sample through the microscope's
observation volume in a microfluidic channel \cite{cheong09}.
Given the camera's exposure time of \SI{0.1}{\ms}, results
are immune to motion blurring for flow rates up to
\SI{200}{\mu m\per\second} \cite{cheong09,dixon11}.
Under typical conditions, no more than ten particles pass through the \SI{86 x 65}{\mu m} field of view at a time.
These conditions simplify the holographic analysis by minimizing overlap
between individual particles' scattering patterns.

Each particle typically is recorded in multiple video frames
as it moves through the field of view.
Such sequences of measurements are linked into trajectories
using a maximum-likelihood algorithm \cite{crocker96}
and median values are reported for each trajectory \cite{cheong09}.
These considerations establish an upper limit to the range
of accessible particle concentrations of \SI{e8}{particles\per\mL}.
At the other extreme, \SI{10}{\minute} of data suffices to
detect, count and characterize particles at concentrations as
low as \SI{e3}{particles\per\mL}.

\section{Lorenz-Mie fitting}
The incident beam may be described as a plane wave, 
\begin{equation} 
\label{incident}
\vec{E}_0(\vec{r}) = u_0(\vec{\rho}) \ \exp(i k z)\ \uvec{\epsilon},    
\end{equation}
which is linearly polarized in the $\uvec{\epsilon}$ direction and propagates with wave number $k = 2\pi n_m / \lambda$ along 
the $\uvec{z}$ direction, where $\lambda$ is the light's wavelength in vacuum and $n_m$ is the refractive index of the medium. For simplicity, we assume that the amplitude profiles $u_0(\vec{\rho})$ at position $\vec{\rho} = (x,y)$ transverse to the optical axis does not depend on axial position $z$.

\begin{figure}[b!]
  \centering
  \includegraphics[width=0.7\columnwidth]{field}
  \caption[Principle of holographic video microscopy]{A colloidal particle scatters a portion ${\boldsymbol{E}}_{s}({\boldsymbol{r}}$,${\boldsymbol{r}}_{p})$ of an initially collimated laser beam ${\boldsymbol{E}}_{0}({\boldsymbol{r}})$. The scattered beam, here interferes with the unscattered portion of the incident beam in the focal plane of a microscope objective, thereby forming an in-line hologram, $I({\boldsymbol{\rho}})$.}
  \label{fig:field}
\end{figure}

The particle at $\vec{r}_p$ scatters a portion of the incident field into the scattered wave $\vec{E}_s(\vec{r})$.
Assuming the object to be small compared with the typical length scale for variations in the illumination, the scattered field may be approximated as
\begin{align} 
\label{scatteredwave}
\vec{E}_s(\vec{r}, \vec{r}_p) = E_0(\vec{r}_p) \ 
\vec{f}_s(k(\vec{r} - \vec{r}_p)),
\end{align}
where $\vec{f}_s(k \vec{r})$ is the scattering function \cite{bohren83,barber90,mishchenko02},
which describes the vector outgoing field scattered by a sphere that is illuminated by a plane wave. The function also depends on the particles' size relative to the wavelength of light, $k a_p$, and on its refractive index to that of the medium, $m = n_p/n_m$.
Fig.~\ref{fig:field} shows schematically the light scattered by the colloidal particle interferes with the rest of the incident field, as the reference beam, to create a hologram,
\begin{align}
\label{interference}
   I(\vec{\rho}) = \abs{\vec{E}_s(\vec{r}, \vec{r}_p) +
                   \vec{E}_0(\vec{r})}^2.
\end{align}

Lorenz-Mie theory allows us to calculate the exact scattered field due to a single isotropic and homogeneous dielectric sphere \cite{bohren83,barber90}.
It is based on matching incident, scattered, and internal fields at the surface of the sphere with appropriate boundary conditions. Lorenz-Mie theory takes advantage of the spherical symmetry of the particle by expanding the fields in a series of the vector spherical harmonics (VSHs),
which is the natural basis for spherically symmetric fields \cite{bohren83,gouesbet11,mishchenko14}. This greatly simplifies the boundary conditions at the expense of making the description of the fields more complicated.
The scattered function can similarly be expanded as \cite{gouesbet10}
\begin{equation}
\label{eq:scatteredfield}
  \vec{f}_s(k \vec{r}) = \sum_{n=1}^\infty \, f_n \, \left(
    i a_n \, \vec{N}^{(3)}_{e1n}(k \vec{r}) - b_n \,
    \vec{M}^{(3)}_{o1n}(k \vec{r})
    \right),
\end{equation}
where $f_n=i^n (2n+1)/[n(n+1)]$, and $\vec{M}^{(3)}_{o1n}(\vec{x})$ and 
$\vec{N}^{(3)}_{e1n}(\vec{x})$ are the vector spherical harmonics,
\begin{equation}
    \vec{M}^{(3)}_{o1n}(\vec{x}) = \frac{\cos\phi}{\sin\theta} \,
  P^1_n(\cos\theta) \, j_n(x) \, \uvec{\theta}
  - \sin\phi \, \frac{dP^1_n(\cos\theta)}{d\theta} \, j_n(x) \, \uvec{\phi},
\end{equation}
and
\begin{multline}
  \vec{N}^{(3)}_{e1n}(\vec{x}) = n(n+1) \, \cos\phi
  \,P^1_n(\cos\theta) \, \frac{j_n(x)}{x} \, \uvec{r} \\
  + \cos\phi \, \frac{dP^1_n(\cos\theta)}{d\theta} \,
  \frac{1}{x} \frac{d}{dx}\left[x j_n(x)\right] \, \uvec{\theta} \\
  - \frac{\sin\phi}{\sin\theta} \, P^1_n(\cos\theta) \,
  \frac{1}{x}\frac{d}{dx}\left[x j_n(x)\right] \, \uvec{\phi}.
\end{multline}
Here, $P^1_n(\cos\theta)$ is the associated Legendre polynomial of the
first kind, and $j_n(kr)$ is the spherical Bessel function of the
first kind of order $n$.
The expansion coefficients in Eq.~(\ref{eq:scatteredfield}) are given
by \cite{bohren83}
\begin{equation}
  a_n = \frac{m^2 j_n(mka_p) \left[ka_p \, j_n(ka_p)\right]^\prime -
    j_n(ka_p) \left[mka_p \, j_n(mka)\right]^\prime}{
    m^2 j_n(mka_p) \left[ka_p \, h^{(1)}_n(ka_p)\right]^\prime -
    h^{(1)}_n(ka_p) \left[mka_p \, j_n(mka_p)\right]^\prime},
\end{equation}
and
\begin{equation}
\label{eq:bn}
  b_n = \frac{j_n(mka_p) \left[ka_p \, j_n(ka_p)\right]^\prime -
    j_n(ka_p) \left[mka_p \, j_n(m ka_p)\right]^\prime}{
    j_n(mka_p) \left[ka_p \, h^{(1)}_n(ka_p)\right]^\prime -
    h^{(1)}_n(ka_p) \left[mka_p \, j_n(mka_p)\right]^\prime},
\end{equation}
where $h^{(1)}_n(x)$ is the spherical Hankel function of
the first type of order $n$, and where primes denote derivatives
with respect to the argument. The sum in Eq.~(\ref{eq:scatteredfield}) converges after a number of terms, 
$n_c = ka_p + 4.05 \, (ka_p)^{1/3} + 2$, which depends on the particle's size \cite{barber90,mishchenko02}.

Once we have calculated the scattered field we can use it to compare with the experimental image. 
We assume that the intensity $I(\vec{\rho'})$, measured at position $\vec{\rho'}$ in the plane of camera is related to the intensity $I(\vec{\rho})$ in the focal plane by the scale transformation $\vec{\rho'} = M \vec{\rho}$ where $M$ is the microscope's lateral magnification. Comparing the recorded pixel array with Eq.~\ref{interference} then involves no more than a change of scale.
To value this comparison, we also subtract off the camera's dark counts, $I_{dc} (\rho')$ from the recorded image, $I_{exp} (\vec{\rho'})$. Dark count is the reading of the camera in the absence of light. It is around 31 for all pixels of the 8-bit NEC TI-324AII camera. In practice, we measure the dark count mask for the camera by recording a short video with no lights on and calculating the median value for each pixel. 

Defects in the optical path create background interference fringes due to the long coherence length of the collimated laser beam. These unwanted features can be minimized by normalizing the raw image with a background image, $I_{bg} (\vec{\rho'})$, that includes no foreground features. The background is usually  obtained by taking the median value of each pixel in a short video, in which particles move within a fixed field of view and a steady environment. It is better to update the background periodically for a long video to keep up with any changes in the background. The image after normalization is ideally free of static background features, 
\begin{equation}
\label{normalize}
   I_{norm} (\vec{\rho'}) = \frac{I_{exp} (\vec{\rho'}) - I_{dc} (\vec{\rho'})}{I_{bg} (\vec{\rho'}) - I_{dc} (\vec{\rho')}} = \frac{I_0 (\vec{\rho})}{\abs{u_0(\vec{\rho})}^2}
\end{equation}
where
\begin{multline}
  I_0 (\vec{\rho}) = \abs {u_0(\vec{r}) \uvec{\epsilon} + u_0(\vec{r_p}) e^{- i k z_p} 
  \vec{f}_s(k(\vec{r} - \vec{r}_p))} ^2\\
  \approx \abs{ u_0(\vec{r}) }^2 \abs{ \uvec{\epsilon} + e^{- i k z_p} 
  \vec{f}_s( k(\vec{r}- \vec{r}_p) ) } ^2
\end{multline}
\begin{figure}[b!]
  \centering
  \includegraphics[width=0.6\columnwidth]{normalize}
  \caption[Example of raw and processed images]{An example of a raw image taken from our holographic microscopy system and the normalized image after dividing the background.}
  \label{fig:norm}
\end{figure}
The normalized image is approximately the magnified version of the normalized light field at focal plane in the field of view, which means
\begin{multline}
\label{LMfit}
   I_{norm} (\vec{\rho'})
   =\abs{\uvec{\epsilon} + e^{- i k z_p} \vec{f}_s(k(\vec{r}- \vec{r}_p))} ^2\\
   	=1 + 2 \, \text{Re}\left\{\vec{f}_s(k(\vec{r}-\vec{r}_p)) \cdot 
   \uvec{\epsilon} \, e^{- i k z_p}\right\} + 
   \abs{\vec{f}_s(k(\vec{r}-\vec{r}_p))}^2,
\end{multline}
Fig.~\ref{fig:norm} is an example of how background normalization can improve the quality of the experimental image.

\begin{figure}[b!]
  \centering
  \includegraphics[width=1.0\columnwidth]{fit}
  \caption[Example of fitting a normalized image]{An example of fitting a normalized image to Lorenz-Mie theory. The violet dashed line and shaded area represent the azimuthal median and standard deviation of the radial profile of the normalized experimentally measured hologram, respectively. The dash-dotted green curve is the radial profile of the theoretical fit to the experiment data. The two holograms of experimental data and theoretical fit are on the right.}
  \label{fig:fit}
\end{figure}

The processed image fits to the theory, yielding useful information of the particle's 3-dimensional position $\vec{r_p}$, size ${a_p}$, refractive index ${n_p}$ and even morphology and surface geometry \cite{hannel15, wang16, wang16a}. We
use the MPFIT \cite{markwardt09} implementation of the Levenberg-Marquardt non-linear least-squares fitter \cite{more78} to find the parameters which best fit the experimental data. Fig.~\ref{fig:fit} shows a typical Lorenz-Mie fit to the measured hologram of a silica particle with $1.4 \mu m$ in diameter.The fitting results suggest that this particular particle has radius $a_p = 0.698 \pm 0.003 \mu m$, refractive index $n_p = 1.401 \pm 0.002$, and is above the focal plane by $z_p = 16.60 \pm 0.01 \mu m$.

\section{Unattended feature identification}
There is another step involved in the Lorenz-Mie fitting procedure, which is feature 
identification. The normalized image usually subtends \SI{640 x 480}{pixel}, and may 
contain none, one or several holograms of particles. We first find the center of each 
holographic feature, and then crop the image around this center to obtain a aquare region 
of interest. The actually fitting process operates on that region.

%\begin{figure}[h!]
%  \centering
%  \includegraphics[width=1.0\columnwidth]{x2}
%  \caption[Detecting particle images in a video hologram.]{Detecting particle images in a video hologram. Original (a) and transformed (b) holographic images of three colloidal spheres. Superimposed line segments in (a) indicate the votes cast by three representative pixels. Intensity in (b) is scaled by the number of votes, with black representing 0 and white representing 800 votes. Superimposed surface plots illustrate the middle sphere’s transformation. Scale bar indicates \SI{10}{mu m}. (Source: Fig. 3 of Ref. \cite{cheong09a})}
%  \label{fig:feature1}
%\end{figure}

Each sphere appears in a snapshot as concentric bright and dark rings. The gradient of the intensity at each pixel, therefore, defines a line segment in the imaging plane along which a sphere's center may lie. The intersection of such lines defines an estimate for the sphere's centroid in the focal plane. Such intersections are identified with a simplified variant of the circular Hough transform \cite{duda72} in which each pixel in the original image casts ``votes" for the pixels in the transformed image that might be centroids. The single-pixel votes are accumulated in a transformed image. 
With the same resolution and number of pixels as the original. The choice yields both reasonable accuracy and speed. Those pixels in the transformed image with the most votes are taken to be centroid candidates, and their locations are used as the in-plane coordinates to initialize fits. Refining the centroid estimate by computing the brightness-weighted center of brightness for each feature in the transformed image typically identifies particles' centroids to within a few tenths of a pixel \cite{crocker96}, or a few dozen nanometers. This is sufficiently precise to ensure convergence of the least-squares fit to the predictions of Lorenz-Mie theory.

\section{GPU acceleration}
\label{sec:GPU}
Fitting image data to the results of Lorenz-Mie theory is computationally
intensive, and requires initial estimates for the adjustable parameters. Holographic fitting lends itself to parallel processing on the graphical processing unit (GPU) of a computer graphics card \cite{owens07}. A sphere's scattering pattern typically subtends tens of thousands of pixels and must be iterated dozens of times in the course of each fit. Each scattering pattern, furthermore, is expressed as an expansion in special functions, each of which must be separately computed. Whereas conventional CPU-based implementations operate on each pixel in sequence, a GPU-enabled algorithm operates
on multiple pixels simultaneously, and so is substantially faster.
We analyze holographic images with software developed in the IDL programming language (Harris Geospatial Solutions), taking advantage of the MPFIT suite of Levenberg-Marquardt nonlinear least-squares fitting routines \cite{markwardt09}. We implemented a GPU-enabled computation of $\vec{f}_s (\vec{r})$ using the GPUlib \cite{messmer08} (Tech-X Corp., Boulder, CO) extensions to IDL on an Nvidia GTX 680 graphics card (Nvidia Corp., Santa Clara, CA) installed in the host computer. GPUlib provides access to the underlying CUDA framework (http://www.nvidia.com/cuda/) for mathematical computation on GPUs. When supported by a multi-core CPU, the GPU can process multiple computational threads simultaneously, yielding a proportional increase in processing speed.

\section{Calibrating the holographic characterization instrument}

Holographic characterization relies on four instrumental calibration
parameters: the vacuum wavelength of the laser illumination,
the magnification of the optical train, the dark count of the camera,
and the single-pixel signal-to-noise ratio at the operating
illumination level.
All of these can be measured once and then used for all
subsequent analyses.

The vacuum wavelength of the laser is specified by the manufacturer
and is independently verified to four significant figures using
a fiber spectrometer (Ocean Optics, HR4000).
The microscope's system magnification is measured to four
significant figures using a precision micrometer scale
(Ted Pella, catalog number 2285-16).
The camera's dark count is measured by blocking the laser
illumination and computing the median reading at each
pixel.
Image noise is estimated from holographic images with
the median-absolute-deviation (MAD) metric.

In addition to these instrumental calibrations, obtaining accurate
results also requires an accurate value for the refractive index of
the medium at the laser wavelength and at the sample temperature.
For the aqueous buffers in the present study, this value was obtained
to four significant figures with an Abbe refractometer (Edmund
Optics).  Approximating this value with the refractive index of
pure water, $n_m = \num{1.340}$, at the wavelength of \SI{447}{\nm} 
and at the measurement temperature
of \SI{21 \pm 1}{\degreeCelsius} yields systematic errors in the estimated radius and refractive index of no more than
\SI{0.1}{\percent}.

\section{Operating range of holographic characterization}

The operating range of the holographic characterization instrument is 
established by measurements on aqueous dispersion of
colloidal spheres intended for use as particle size standards.
The interference fringes in each particle's holograms must
be separated by at least one pixel in the microscope's focal plane.
This requirement is accommodated by setting the focal plane at least 
\SI{5}{\mu m} below the particle in the sample cell. The largest accessible axial displacement is set both by the need to fit multiple concentric fringes into the camera's field of view, and also by the reduction of image contrast below the camera's noise floor. This upper limit is roughly \SI{50}{\mu m} for this instrument.

The lower end of the range of detectable particle sizes is
limited to half the wavelength of light in the medium.
Particles smaller than this yield detectable holograms, which can be fit by Lorenz-Mie theory.  These fits, however, do not cleanly separate the particle size from the refractive index.  If the particle's
refractive index is known \emph{a priori}, these measurements
again can yield reliable estimates for the particle's radius.
For the present measurements, the practical lower limit is set by the 8-bit dynamic range of the camera to $a_p \gtrsim \SI{200}{\nm}$. Smaller particles' light-scattering patterns lack the contrast needed for reliable detection and characterization.

The upper size limit is set to $a_p \lesssim \SI{10}{\mu m}$ by the depth of the channel, separation of the fringes and the number of fringes.  We do not, moreover, expect to observe and correctly identify particles much larger than \SI{10}{\mu m}.


\section{Comparison with established techniques}
\label{sec:comparison}

Among available particle-characterization techniques, holographic characterization is unique in its ability to measure both the
size and the refractive index of individual particles. In other 
respects, it complements established techniques.

As an imaging technique, holographic characterization is related to 
Micro-Flow Imaging (MFI) and Nanoparticle Tracking Analysis (NTA). Holographic characterization benefits, however, from its large effective depth of field and its ability to monitor refractive index as well as size. Because MFI and holographic characterization can characterize a single particle with a single snapshot and so are inherently faster than NTA, which relies on time-series analysis of a particle's Brownian motion to estimate its hydrodynamic radius. This analysis also requires information on the solvent's viscosity, which is not required for MFI or holographic characterization. NTA is best suited for sizing submicrometer particles, which diffuse quickly \cite{filipe10}. MFI is most effective for the large subvisible particles \cite{sharma15}. Holographic characterization bridges the gap.

Holographic characterization also is related to light-scattering
techniques such as Dynamic Light Scattering (DLS) \cite{filipe10}.
Like NTA, DLS is sensitive to smaller particles than holographic 
characterization. As a sample-averaged measurement, however, 
DLS does not inherently account for heterogeneity in sample composition. 
Estimating relative abundances of particles with different sizes,
furthermore, requires information such as the particles' refractive
indexes and the solvent's viscosity to be provided as inputs.
Particle-resolved characterization techniques, including NTA, MFI,
and holographic characterization, therefore provide more reliable
information on size distributions, particularly in heterogeneous
samples. Single-particle methods also are sensitive to lower 
concentrations of particles than DLS.

By measuring the degree to which individual particles block a
beam of light, light obscuration (LO) combines some benefits of
light scattering and particle imaging \cite{demeule10,zolls13}. 
Typically used for 
particles larger than \SI{2}{\mu m}, LO requires particles to be well-enough
separated in the flow to be detected individually. Samples with
concentrations higher than \SI{2e4}{particles/mL} may have to be diluted into this operating range \cite{demeule10}. LO cannot distinguish different types of particles in heterogeneous samples.

Holographic characterization offers greater counting sensitivity for objects of micrometer scale 
than dynamic light scattering, and access to smaller particles
than light obscuration, without requiring dilution.
Unlike other scattering techniques, holographic characterization
does not require the particles' refractive indexes as inputs, but
rather provides the refractive index as an output.

Holographic characterization is orthogonal to non-optical techniques
as the Coulter principle or Resonant Mass Measurement (RMM), both of which are
particle-resolved measurement techniques. Coulter counters yield
measurements of particles' volumes over a size range determined
by the choice of aperture through which the particles flow. Clogging 
can pose problems \cite{demeule10}. Like LO, the Coulter principle requires
particles to pass through the measurement volume one at a time,
which sets an upper limit on particle concentration of 
\SI{e5}{particles/mL}. In addition, Coulter counters 
impose requirements on
the solution's conductivity. Achieving operating conditions may
require changes to samples' chemical environment that could influence 
their properties \cite{demeule10}.

RMM measures the buoyant mass of individual particles by their
influence on the resonant frequency of the microfluidic channel
through which they flow \cite{zolls13,weinbuch13,panchal14}. 
This approach naturally differentiates 
objects by their volumetric mass density. It cannot, however, differentiate materials with the same sign of the buoyant mass. Because it requires
particles to pass through a narrow channel, RMM analyzes smaller
sample volumes per unit time than holographic characterization
and so cannot easily probe sample concentrations below 
\SI{e6}{particles/mL}. As a measurement of buoyant mass, moreover, RMM
does not directly report particle size.

Comparisons among these techniques are summarized
in Table~\ref{tab:comparison} and Table~\ref{tab:comments}.

\setlength{\extrarowheight}{0.5em}
\begin{table}[!t]
  \centering
  \caption*{\bf{Comparison of high-throughput characterization techniques for subvisible colloidal systems}}
  \begin{tabular}{>{\raggedright}p{6cm} c c c} \hline
    {\bf Method} & 
    {\bf Size [\si{\mu m}]} &
    {\bf Number/ml} &
    {\bf Morphology} \\\hline
    Holographic Characterization &
        \numrange{0.3}{10} &
        \numrange{e3}{e8} &
        Yes  \\
    Dynamic Light Scattering (DLS) \cite{filipe10,panchal14} &
        \numrange{0.001}{1} & 
        \numrange{e8}{e12} &
        No  \\
    Electric Sensing Zone (ESZ), 
    Coulter Principle \cite{demeule10} &
        \numrange{0.1}{1600} &
        \numrange{1}{e5} &
        No \\
    Light Obscuration (LO) \cite{demeule10,zolls13} & 
        \numrange{1}{200} & 
        \numrange{e3}{e5} & 
        No \\
    Dynamic Imaging Analysis (DIA), 
    Micro-Flow Imaging (MFI) \cite{zolls13,weinbuch13} &
        \numrange{1}{400} & 
        \numrange{e4}{e6} &
        Yes  \\
    Nanoparticle Tracking Analysis (NTA) \cite{filipe10} &
        \numrange{0.03}{1} & 
        \numrange{e7} {e9} &
        No \\
    Resonant Mass Measurement (RMM),
    Archimedes
    \cite{zolls13,weinbuch13,panchal14} &
        \numrange{0.3}{4} &
        \numrange{e6}{e9} &
        No \\
  \hline
  \end{tabular}
  \caption[Comparison of particle characterization techniques 1]{The size range refers to the radius of the effective sphere detected by each method. The fourth column indicates whether the technique is capable of measuring morphology of particles. References describe independent assessments of techniques' capabilities for characterizing colloidal systems.}
  \label{tab:comparison}
\end{table}

\clearpage

\setlength{\extrarowheight}{0.5em}
\begin{table}[!t]
  \centering
  \begin{tabular}{>{\raggedright}p{6cm} p{7.5cm}} \hline
    {\bf Method} & 
    {\bf Comments} \\\hline
    Holographic Characterization &
        Measures both size and refractive index. Differentiates by refractive index. Can differentiate by morphology. \\
    Dynamic Light Scattering (DLS) &
        Sample-averaged measurement. 
        No differentiation. \\
    Electric Sensing Zone (ESZ), 
    Coulter Principle  &
        Requires compatible electrolyte. 
        Typically requires sample dilution. 
        Requires calibration with size standards. 
        Size range determined by orifice selection. 
        No differentiation.\\
    Light Obscuration (LO)  & 
        Typically requires sample dilution. 
        Sensitive to refractive index variations. 
        Requires calibration with size standards.
        No differentiation.\\
    Dynamic Imaging Analysis (DIA), 
    Micro-Flow Imaging (MFI) &
        Differentiates by morphology. \\
    Nanoparticle Tracking Analysis (NTA) &
        Measurement time increases with particle radius. 
        No differentiation.\\
    Resonant Mass Measurement (RMM),
    Archimedes &
        Particle size estimated indirectly from its mass. 
        Differentiates between positively and negatively buoyant particles.\\
  \hline
  \end{tabular}
  \caption[Comparison of particle characterization techniques 2]{The term ``differentiation'' refers to a technique's ability to distinguish particles of similar size but different composition.}
  \label{tab:comments}
\end{table}

\newpage

\section{Holographic flow velocimetry}

\begin{figure}[t!]
  \centering
  \includegraphics[width=0.7\columnwidth]{flow}
  \caption[Poiseuille flow profile]
  {The velocity measurement of polystyrene tracers in a Poiseuille flow. The flow 
  is slowing down over time. The symbols are colored by time.}
  \label{fig:flow}
\end{figure}

One of the application of holographic microscopy is to holographic flow velocimetry \cite{cheong09}. By tracking the colloidal particles' velocities in a low Reynolds number flow, holographic microscopy is able to reconstruct the flow profile and provide information about the boundary condition of the flow.

Here, we use \SI{0.58}{\um} radius polystyrene particles as tracers to track the Poiseuille flow profile caused by a decaying pressure in a microfluidic channel made by a pair of parallel glass slide and cover slip. the horizontal velocity is calculated from the x-y position in two consecutive frames. We use small polystyrene particle because its gravitational height is long enough to cover the height of the channel.

From Fig.~\ref{fig:flow}, it is clear to see that the flow velocity decreases over time due to the fluid pressure weakened by the balancing of two reservoirs reside on each end of the channel.

Just taking the data from the last two minutes, we are able to fit the Poiseuille flow profile into a parabola function to find out the bottom and the top of the microfluidic channel. The height of the channel can be easily extracted from the fits, which is \SI{22}{\um} in this case.

\begin{figure}[t!]
  \centering
  \includegraphics[width=0.7\columnwidth]{height}
  \caption[Parabolic fit to the Poiseuille flow profile]
  {Parabolic fit to the Poiseuille flow profile from the last two minutes data in 
  Fig.~\ref{fig:flow}. Symbols are colored by the probability density of the data 
  point in the $z-v$ phase space.}
  \label{fig:height}
\end{figure}

The data points are colored by the number density of the data point. The probability density is calculated by kernel density estimator \cite{silverman86}. Each discrete data point is replaced by a continuous kernel function. The probability distribution function in the phase space is the normalized sum of all kernels. Epanechnikov kernel and Gaussian kernel are the most common choices. A histogram uses a uniform kernel function.