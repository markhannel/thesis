\chapter{Holographic Video Microscopy}
\label{ch:hvm}

% Suggested figures.
% Select Historical contributions.
% Diagrams depicting the physical processes
% sequentially.j

\newcommand{\einc}{\vec{E}_{\text{inc}}}
\newcommand{\escat}{\vec{E}_{\text{s}}}
\newcommand{\eadd}{\vec{E}_{\text{add}}}


\section{The Advent of Holography}

% Review of literature and historical perspective.



\section{Holographic Video Microscopy}
\label{ch:hvm:sec:hvm}

Holographic Video Microscopy differs from
traditional microscopy techniques by preserving,
collecting, and making quantitative use of phase
information; this information is
%Traditional microscopy utilizes
%incoherent illumination and therefore all
%phase information is 
\subsection{Broad Overview}
\label{ch:hvm:sec:hvm:ssec:overview}

% The importance of phase information retrieval.

%

% Sample preparation.
% Cleaning and dilution of sample necessary
% due to the depth of FOV.

\begin{eqnarray}
  I &=& \abs{\einc + \escat}^2\\
    &=& \abs{\einc}^2 + \einc\cdot\escat^* + \einc^*\cdot\escat + \abs{\escat}^2\\
    &=& \abs{\einc}^2 + 2\Re\left \{\einc\cdot\escat^*\right \} + \abs{\escat}^2
\end{eqnarray}

\begin{eqnarray*}
  I &=& \abs{\einc + \escat + \eadd}^2\\
    &=& \abs{\einc}^2 + 2\Re\left \{\einc\cdot\escat^*\right \} + \abs{\escat}^2
\end{eqnarray*}


\subsection{Illumination}
\label{ch:hvm:sec:hvm:ssec:illumination}


\subsection{Scattering}
\label{ch:hvm:sec:hvm:ssec:scattering}
  

\subsection{Collection and Refocusing}

\subsection{Digital Recording}
\label{ch:hvm:sec:hvm:ssec:digitalrec}

% What are digital images. How are digital images recorded?
% Why are you telling them such things?
% How is it that digital images do not record intensity?

Images in science are ubiquitous; by sampling the intensity over an exposure
period, a single image encodes a bevy of spatial information that is ripe
for qualitative and quantitative analysis. Historically, researchers would
extract this information by hand. Jean-Perrin used a projector, pen and a paper to originally
measure the diffusion of a polymer bead. Crystallographers such as %FIXME: Add reference.
traced photographs of crystalline diffraction patterns by hand. Presently researchers utilize
digital cameras to record and digitize images as arrays of values. 

We utilize a digital camera to record holograms. To fit our experimental holograms to
theory and to simulate our digital recordings it is necessary to have a basic
understanding digital images. In this section we will 

\subsubsection{Digitizing Images}
\label{ch:hvm:sec:hvm:ssec:digitalrec:sssec:digitizing}

Modern digital cameras employ an array of photon detectors to measure the
average intensity over the sensor surface. Each photon detector, referred to as a pixel,
utilizes the photoelectric effect to convert photons into excited electrons.
The excited electrons at each pixel are counted, sometimes to single precision, and
then digitized into $8$-, $12$-, or even $16$-bit integers. Because the number of
excited electrons is proportional to the number of photons, and the the number of
photons is in turn proportional to the intensity of the of the image at the
sensor surface, the array of electron counts serves as a proxy for the intensity
of the image.

With millions of pixels and the capacity to count tens of thousands of electrons at each pixel,
digital cameras are an engineering marvel. For our purposes, we will review a few of the physical
details of a single pixel to explain important imaging effects such as dark counts, saturation,
and the digitization procedure.

Pixels are designed to accurately count the number of electrons that are
excited by incident photons within a given exposure period. A number of physical constraints
limit the accuracy of each pixels.

During an exposure period, $N_p$ photons of a particular wavelength arrive at a pixel
surface. Some fraction of the incident photons are converted to excited electrons
with a wavelength dependent probability known as the quantum efficiency. To be properly
counted, these excited electrons must survive until the counting procedure has
accounted for their presence. To this end, each pixel is doped to increase the lifetime
of excited electrons. In addition, the excited electrons must remain in the bulk so that
they are not grounded; for this purpose, a biased field is applied. % FIXME: What about mirror charges?

During the exposure period, a number of electrons can be thermally excited (as opposed to
photonically excited) and will be included in the count. These erroneously
counted electrons are referred to as a dark count. Measuring the average dark count of a
camera is as easy as recording the average image while an opaque object is blocking the
camera's sensor. Note that the dark count will increase with the length of the exposure
period.

% The number of electrons that can be negative.. scientific cameras have a non-zero
% floor to maintain gaussian-errors.
% Saturation occurs because the relation number of excited electrons per
% number of incident photons becomes non-linear. The largest number of reported
% electrons is the highest level


\section{The Inverse Problem}

\subsection{Feature Detection}

After appropriately normalizing an image, it is necessary to detect if any holographic
features are in the field of view. 

\subsection{Characterizing Holograms}

\section{Applications of HVM}
