\chapter{Accounting for the vectorial nature of light propagation}
\label{ch:debye}



%Infinity corrected objectives enable a robust range of placements for the
%tube lens relative to the back aperture of the objective without noticeable
%loss in image quality. 

\section{Introduction}

The model for image formation outlined in \autoref{ch:hvm} provides a full
vectorial treatment of the propagation of the illuminating plane wave and
the scattered resulting scattered fields
up to the focal plane of the objective. From there, it is assumed that the
objective and tube lens simply magnify and relay the intensity pattern from
the focal plane to the camera plane.
This is a base assumption of traditional microscopy and is not unfounded; optics
manufacturers are economically incentivized to produce quality optics which
accurately image the specimen plane. However, these optics are optimized for
incoherent imaging of specimens in the focal plane;
how well does this venerable assumption hold for coherent imaging of out-of-plane
scatterers?

In this chapter, we will account for the propagation of the electric fields from
the specimen plane to the imaging plane. Our model includes the
refraction, reflection, and angular demagnification of the fields as they propagate
through the objective and tube lens before arriving at the camera plane.
We will attempt a full vectorial treatment throughout and will
validate any scalar wave approximations made along the way. In many ways our work
parallels and extends the work of Izen and Ovyrn\cite{izen00} and the work of
\c{C}apo\u{g}lu\cite{capoglu12} et al. In particular, we adopt follow the format of the
former and adopt the mathematical conventions and machinery of the latter.
While our work primarily serves to validate the scalar theory
presented in \autoref{ch:hvm}, we probe any physical limits
of HVM set by collecting, refocusing, and recording the confluence of the incident
beam and scattered wave.


\section{Modeling the Optical Train}

%% %FIXME maybe provide a super BRIEF recap of LM up to the focal plane?
%Figure 1 represents the physical situation underlying our experimental technique.
%A colloidal sphere with radius $a_p$ and refractive index $n_p$ is contained
%in a volume of fluid with refractive index $n_m$. The dilute colloidal suspension
%is contained between a coverslip and glass slide of refractive index $1.5$
%(for now we neglect the effect of the coverslip). An incident wave of coherent
%illumination stimulates the colloidal sphere into radiating a spherical wave
%described by the Lorenz-mie theory. The scattered wave and incident wave continue
%propagating towards the camera plane.

The optical train for our holographic microscope is quite simple.
Besides a coherent illumination source, our setup can be reduced to the four
main components of conventional microscopy: the sample, an objective,
a tube lens, and a camera. In the previous chapter, we described the
scattering of a plane wave by a spherical scatterer up to the focal plane
of the objective. We will now evaluate the scattered field and the plane
wave at several interfaces along the optical axis. Referring to
Figure %FIXME
we will perform the following calculations:
\begin{enumerate}
\item Evaluate the scattered field at the entrance pupil of the objective.
\item Account for the  $z$ displacement of the particle away from the focal
  plane of the objective.
\item Compute the scattered field at the exit pupil of the objective by accounting
  for the angular demagnification imposed by the objective.
\end{enumerate}


A distance $z_p + f$ away from the colloidal sphere, the two waves enter the
objective pupil. The objective relays the two waves to the tube lens through
infinity space. The tube lens refocuses the waves such that they arrive at the imaging
plane magnified.

Our optical model makes use of the debye-wolf integral formalism. By computing the
angular spectrum present at the tube lens plane, we compute the electric fields
present at the imaging plane. Owing to the linearity of our optical system,
we consider the incident field and scattered field separately. We summarize the physical
steps necessary for each wave before discussing the details in section 3 and section 4.

Since the incident field propagates parallel to the optical axis, we will assume that
the incident field is only laterally magnified and phase shifted as it traverses the
optical system.
\begin{enumerate}
\item[1.] Account for the phase displacement on the object side of the objective.
\item[2.] Account for the position independent lateral magnification.
\item[3.] Account for the phase displacement on the image side of the tube lens.
\end{enumerate}
The scattered wave does not propagate parallel to the optical axis and is therefore
battered by the objective and tube lens. We will assume the objective is not a
piece of shit and obeys the abbe-sine condition. In addition, we will trust that the
objective obeys the conservation of energy for all rays which enter it's pupil.
\begin{enumerate}
\item[1.] Compute the angular spectrum of the scattered field on a spherical surface
  in the far-field.
\item[2.] Apply the apodization caused by the entrance pupil of the objective.
\item[3.] Account for the direction dependent magnfication. This induces changes in
  amplitude, phase and direction (thus polarization).
\item[4.] Compute the electric fields present at the imaging plane from the angular spectrum in step 3.
\end{enumerate}

Finally the resulting image is computed with via the poynting vector integrated over pixels.

\begin{equation*}
  I = \left | E_{inc} + E_{s} \right |^2 
\end{equation*}


  {\bf Propagating the Reference Wave.}\\ 
  The reference wave is assumed to be traveling coaxially through the optical
  train. It is collected by the objective, passed through ``infinity'' space
  and then rectified back to a plane wave. In the process, the field is flipped 
  (a $\pi$-shift in phase) but the polarization remains. 
  To maintain the conservation of energy, we should measure the flux of energy 
  a single, infinitesimal patch of the wave as it enters and exits the optical 
  train (see figure 1).

  \begin{eqnarray*}
    I_1 dA_1 &=& I_2 dA_2 \\
    \frac{n_1}{\eta_o} \left | \vec{E}_1 \right |^2 dA_1 &=& \frac{n_2}{\eta_o} \left | \vec{E}_2 \right |^2 dA_2\\
    \frac{n_1}{n_2} \left | \vec{E}_1 \right |^2 \frac{dA_1}{dA_2} &=& \left | \vec{E}_2 \right |^2 \\
    \frac{n_1}{n_2}  \frac{1}{M^2} \left | \vec{E}_1 \right |^2 &=& \left | \vec{E}_2 \right |^2 \hspace{.25 in}\text{magnification by M in -x,-y}\\
  \end{eqnarray*}

  By including the $\pi$-shift, we arrive at a final result for the plane wave:
  \begin{equation*}
    \vec{E}_2 = -\frac{1}{M}\sqrt{\frac{n_1}{n_2}} \vec{E}_1     
  \end{equation*}

  Until image formation, the following steps will exclusively refer to the 
  scattered field.
   {\bf Scattering.}\\
  The reference wave (before traversing the optical train) illuminates the 
  spherical scatterer. The spherical scatterer emits a spherical wave
  as a result with some of it's physical properties encoded in the wavefront
  (namely it's radius, refractive index and position). 

  We can imagine the spherical wave as a surface expanding radially away
  from the scatterer. When the radial distance $r$ is large enough, the
  spherical wave's $r$ dependence takes the following functional form:
  \begin{equation*}
    \vec{E}(r, \theta, \phi) = \frac{\vec{E}(\theta, \phi)e^{ikr}}{r}
  \end{equation*}
  The term $\vec{E}(\theta, \phi)$ is known as the strength factor and it
  has only a $\phi$ and $\theta$ component (any component in $r$ will attenuate
  with distance). In what follows, the strength factor will be easier to work
  with. To this end, we've modified spherical\_field.py to accept a keyword 
  argument str\_factor which when evaluated as True, spherical\_field returns 
  the strength factor rather than the electric field.

  At this point we should note two things. First, the electric field strength
  has units of $ \left [ \vec{E}(\theta, \phi) \right ] = \text{volts}$, and not 
  volts per meter (this will save some people from later confusion). Secondly,
  we've tied ourselves to an assumption. 
  {\bf Whenever the scattered field cannot
  suitably be cast in the above form, our formalism will break down.}

  For the purposes of our 
  {\bf Displacement.}\\
  The scattered field above gives an electric field strength factor that
  emanates from the origin of it's coordinate system. In the eyes of the
  collection step, this will mean that the scatterer is located at the origin
  of the focal plane. The scatterers we hope to investigate will be staged 
  some distance z above the focal plane such that the interefence patterns 
  culiminating in the camera plane will be maximally-information dense.

  To account for the z-displacement of the electric field, we will consult 
  John Goodman's ``Introduction to Fourier Optics''. In section 3.10.2, he 
  details
  how one would propagate an electric field strength factor between parallel
  planes. In our case, we hope to displace the 
  angular spectrum from the focal plane to a plane parallel to the focal plane
  but containing the scatterer. In Goodman's text\cite{goodman05}, this would be a negative
  distance and yet in our geometry we have defined it to be a positive distance.
  Therefore we adopt Eq 3-66 of Goodman with the alteration that  $z \rightarrow -z$.
  \begin{equation*}
    \vec{E}(\theta, \phi)|_{\text{scatter plane}} = \vec{E}(\theta, \phi)|_{\text{focal plane}} e^{-i\frac{2\pi}{\lambda}z\cos{\theta} }
  \end{equation*}

  Note, we will later adopt the direction cosines $(s_x, s_y)$ used in Goodman.
  {\bf Collection.}\\
  Here we will repeat the conservation of energy arguments present in section
  1, but unlike the plane wave, we will find that the amplitude and 
  polarization of the spherical wave is perturbed by the optical train in a 
  $\theta$ dependent fashion. 

  \begin{equation*}
    \begin{split}
    \vec{E'}(\theta', \phi')\cdot\hat{\theta'} & = - M \sqrt{ \frac{n'\cos{\theta'}}{n\cos{\theta}}}\vec{E}(\theta, \phi)\cdot\hat{\theta}\\
    \vec{E'}(\theta', \phi')\cdot\hat{\phi'} & = - M \sqrt{ \frac{n'\cos{\theta'}}{n\cos{\theta}}}\vec{E}(\theta, \phi)\cdot\hat{\phi}
    \end{split}
  \end{equation*}
{\bf Refocusing.}\\
  The Debye-Wolf integral provides a way to propagate an electric field strength
  factor from one plane to another via an integral formalism. It is given 
  as:

  \begin{equation*}
    \vec{E}_{img}(x', y') = \frac{i k'}{2 \pi} \iint_{\Omega_{img}} \vec{E'}_s(s'_x, s'_y) e^{-ik'(s'_xx'+s'_yy')}\frac{ds'_xds'_y}{\cos{\theta'}}
  \end{equation*}

  There are many things worth mentioning:
  \begin{enumerate}
  \item The Debye Wolf integral takes a field strength factor $\vec{E'}_s$,
    an integration domain $\Omega_{img}$ and the wavenumber $k'$ as it's inputs.
    From these inputs, the electric field in the (x'-y') plane is
    calculated. 
  \item Heuristically, one should imagine that $\Omega_{img}$ dictates how far 
    away the plane (x'-y') will be (the distance scales with the size of 
    $\Omega_{img}$), that $k'$ gives the field strength factor the necessary 
    units to become an electric field and the electric field strength factor 
    contains most of the information detailing the interference pattern in the
    x'-y' plane ($k'$ has influence here as well).
  \item This integral in many ways amounts to
  an inverse of Huygen's rule; a spherical wave is decomposed into a multitude 
  of plane waves whose phase relations and subsequent inteference produce
  the same effect as the original spherical wave.
  \item The Debye-Wolf formalism easily accounts for any phase aberration 
    $\Phi(s'_x,s'_y)$ of the field as it passes through the optical train. To do so, the integrand
    of the Debye-Wolf integral is multiplied by a phase term $e^{ik'\Phi(s'_x,s'_y)}$.
  \end{enumerate}

  With the exception of the $k'$ in the phase term and the $\cos{\theta'}$ 
  modifying the differential element, we should recognize the
  Debye-Wolf integral as a fourier transform. We will discretize this
  fourier transform subject to sampling and aliasing conditions described
  in the appendix. I will simply summarize the reference's result and then
  derive the minor cosmetic changes present in debyewolf.

  \begin{equation*}
    \begin{split}
      \vec{E}_{img}( m \Delta_x, n \Delta_y) & \approx \frac{i k' \Delta s'_x \Delta s'_y}{2 \pi} e^{-i2\pi \left ( \frac{s'_{x_o}}{\Delta s'_x N_p} m + \frac{s'_{y_o}}{\Delta s'_yN_q} n \right ) } \vec{E}\left [ m, n \right ] \\
      \vec{E}\left [ m,n \right ] & = \sum_{p=0}^{N_p-1}\sum_{q=0}^{N_p-1}\vec{G'}\left [p,q\right ] e^{-i2\pi \left ( \frac{pm}{N_p}+\frac{qn}{N_q} \right ) } \\
      \vec{G'}(s'_x,s'_y) & \doteq \frac{\vec{E'}_s(s'_x,s'_y)}{\cos{\theta'}}e^{-ik'\Phi(s'_x,s'_y)}
    \end{split}
  \end{equation*}
  
  

  Let's first reduce the prefactor in $\vec{E}_{img}$:
  \begin{eqnarray*}
    \frac{-ik'\Delta s'_x \Delta s'_y}{2 \pi} &=& -i\left ( \frac{k'}{2\pi} \right ) (2 \sin{\theta_{img}}/P)(2 \sin{\theta_{img}}/Q) \\
    &=& -i \left ( \frac{1}{\lambda'} \right ) \frac{4}{PQ} \left ( \sin{\theta_{img}} \right )^2 \hspace{.5 in} \text{By Equation 132. of Ref.}\\
    &=& -i \left ( \frac{1}{\lambda'} \right ) \frac{4}{PQ} \left ( \frac{\text{NA}}{Mn'} \right )^2 \\
    &=& -i \left ( \frac{4}{PQ} \right ) \left ( \frac{\text{NA}^2}{M^2\lambda n'} \right )
  \end{eqnarray*}

Now reduce the exponential in $\vec{E}_{img}$:
\begin{eqnarray*}
  i2\pi \left ( \frac{s'_{x_o}}{\Delta s'_x N_p} m + \frac{s'_{y_o}}{\Delta s'_yN_q} n \right )  &=& i2\pi \left ( \frac{-\sin{\theta_{img}} \left ( 1 - 1/P\right )}{2\sin{\theta_{img}}/P\cdot N_p} m + \frac{-\sin{\theta_{img}}\left ( 1-1/Q\right ) }{2\sin{\theta_{img}}/Q \cdot N_q} n \right ) \\
  &=& -i\pi \left ( \frac{P - 1}{N_p}m + \frac{Q - 1}{N_q} n \right ) 
\end{eqnarray*}
Therefore our expression for $\vec{E}_{img}$ is the following:

\begin{equation*}
  \vec{E}_{img} \approx -i \left ( \frac{4}{PQ} \right ) \left ( \frac{\text{NA}^2}{M^2\lambda n'} \right ) e^{-i\pi \left ( \frac{P - 1}{N_p}m + \frac{Q - 1}{N_q} n \right ) }\vec{E}\left [ m, n \right ]
\end{equation*}


  
  In this work we've sought to produce an image in the camera plane from the
  fields present in the particle plane. To achieve such a transformation,
  we've treked through the optical train and accounted for various 
  effects including energy conservation, angular demagnification, polarization
  rotation, lensing effects and image formation. In doing so, we've
  implemented the Debye-Wolf formalism via a 2D fourier transform. To ensure that
  our sampling rate will not cause aliasing, we impose the following condition on
  the sampling number $P$ (the same argument applies to the sampling number $Q$):
  \begin{eqnarray*}
    \Delta s_x &<& \frac{2 \pi}{k W_x} \hspace{.5 in} \text{Eq 142 in Ref.}\\
    \frac{2 \sin{\theta_{img}}}{P} &<& \frac{\lambda}{W_x} \\
    \frac{2 \sin{\theta_{img}}W_x}{\lambda} &<& P \\
    \frac{2 \text{NA}\cdot W_x}{n\lambda} &<& P 
  \end{eqnarray*}
  
  The field $\vec{E}_{img}$ is discretized into a set of points 
  $\left ( \Delta_x, \Delta_y \right )$ in the camera plane. We desire that the
  spacing in the image approximates the real spacing between adjacent 
  pixels of our camera. As such, we seek to satisfy the relation 
  $\Delta_x = M\text{mpp}$. However, this is an integer equation which can only
  be approximately satisfied in general. Pursuing this approximation:
  \begin{eqnarray*}
    \Delta_x &\approx& \text{mpp}\cdot M \\
    \frac{2 \pi}{k' \Delta s_x' N_p} &\approx& \text{mpp}\cdot M \\
    \frac{\lambda'}{N_P} \frac{1}{\Delta s_x'} &\approx& \text{mpp}\cdot M \\
    \frac{\lambda'}{N_P} \frac{P}{2 \sin{\theta_{img}}} &\approx& \text{mpp}\cdot M \\
    \frac{\lambda'}{N_P} \frac{PMn'}{2\text{NA}} &\approx& \text{mpp}\cdot M \\
    \frac{\lambda}{N_P} \frac{P}{2\text{NA}} &\approx& \text{mpp} \\
    \frac{P\lambda}{2\text{NA}\cdot\text{mpp}} &\approx& N_p \\
    \frac{P\lambda}{2\text{NA}\cdot\text{mpp}} &\approx& P+\text{pad}_P \\
    \left ( \frac{\lambda}{2\text{NA}\cdot\text{mpp}} - 1 \right )P &\approx& \text{pad}_P \\    
  \end{eqnarray*}

  \begin{equation*}
    \begin{split}
      P > \frac{2 \text{NA}\cdot W_x}{n\lambda} \\
      \text{pad}_P  \approx  \left ( \frac{\lambda}{2\text{NA}\cdot\text{mpp}} - 1 \right )P
    \end{split}
  \end{equation*}


  
\section{Comparison to Scalar Diffraction Theory}

More work.

\section{Discussion}

Not much to discuss yet.


