\chapter{Accounting for the Vectorial Nature of Light Propagation}
\label{ch:debye}



%Infinity corrected objectives enable a robust range of placements for the
%tube lens relative to the back aperture of the objective without noticeable
%loss in image quality. 


\section{Introduction}


%The Lorenz-Mie Scattering theory has been applied in a host of contexts:
%dust particles in space, dynamic and static light scattering, color spectra
%through mists, etc.

%The structures and dynamics of soft matter systems are often micrometer in 
%scale. In addition, smaller features of interest can be examined with micrometer
%scale probes. Since visible wavelengths are on the order of a micrometer, such 
%systems are ripe to be investigated with visible wavelength microscopy 
%techniques.

%Conventional microscopy has been ubiquitously used to investigate soft matter systems. However, bright field illumination provides amplitude-only information as phase information has been averaged away. Some phase-based techniques are simply used to increase contrast in images. However, techniques like Holographic Video microscopy make full use of phase information contributing to image formation at the camera plane.

The model for image formation outlined in \autoref{ch:hvm} provides a full
vectorial treatment of the propagation of the incident beam and the scattered
wave up to the focal plane of the objective. From there, it is assumed that the
objective and tube lens simply magnify and relay the intensity pattern from
the focal plane to the camera plane.
This is a base assumption of traditional microscopy and is not unfounded; optics
manufacturers are economically incentivized to produce optics which accurately image
the specimen plane. However, these optics are optimized for incoherent imaging of
specimens in the focal plane;
how well does this venerable assumption hold for coherent imaging of out-of-plane
scatterers?

In this chapter, we will account for the propagation of the electric fields from
the specimen plane to the imaging plane. Our model includes the
refraction, reflection, and angular demagnification of the fields as they propagate
through the objective and tube lens before arriving at the camera plane.
We will attempt a full vectorial treatment throughout and will
validate any scalar wave approximations made along the way. In many ways our work
parallels and extends the work of Izen and Ovyrn\cite{izen00} and the work of
\c{C}apo\u{g}lu\cite{capoglu12}. In particular, we adopt follow the format of the
former and adopt the mathematical conventions and machinery of the latter.
While our work primarily serves to justify the scalar wave propagation of the
image through the optical train, we also hope to probe any possible limitations
of HVM set by collecting, refocusing, and recording the confluence of the incident
beam and scattered wave.


\section{Modeling the Optical Train}
Holographic video microscopy takes advantage of coherent illumination in order
to preserve phase information. Besides the choice of illumination, our HVM setup can be reduced to the four components of conventional microscopy: the sample, an objective, a tube lens and a camera.

Figure 1 represents the physical situation underlying our experimental technqiue. A colloidal sphere with radius $a_p$ and refractive index $n_p$ is contained in a volume of fluid with refractive index $n_m$. The dilute colloidal suspension is contained between a coverslip and glass slide of refractive index $1.5$ (for now we neglect the effect of the coverslip). An incident wave of coherent illumination stimulates the colloidal sphere into radiating a spherical wave described by the Lorenz-mie theory. The scattered wave and incident wave continue propagating towards the camera plane.

A distance $z_p + f$ away from the colloidal sphere, the two waves enter the objective pupil. The objective relays the two waves to the tube lens through infinity space. The tube lens refocuses the waves such that they arrive at the imaging plane magnified.

Our optical model makes use of the debye-wolf integral formalism. By computing the angular spectrum present at the tube lens plane, we compute the electric fields present at the imaging plane. Owing to the linearity of our optical system,
we consider the incident field and scattered field separately. We summarize the physical steps necessary for each wave before discussing the details in section 3 and section 4.

Since the incident field propagates parallel to the optical axis, we will assume that the incident field is only laterally magnified and phase shifted as it traverses the optical system.
\begin{enumerate}
\item[1.] Account for the phase displacement on the object side of the objective.
\item[2.] Account for the position independent lateral magnification.
\item[3.] Account for the phase displacement on the image side of the tube lens.
\end{enumerate}
The scattered wave does not propagate parallel to the optical axis and is therefore battered by the objective and tube lens. We will assume the objective is not a piece of shit and obeys the abbe-sine condition. In addition, we will trust that the objective obeys the conservation of energy for all rays which enter it's pupil.
\begin{enumerate}
\item[1.] Compute the angular spectrum of the scattered field on a spherical surface in the far-field.
\item[2.] Apply the apodization caused by the entrance pupil of the objective.
\item[3.] Account for the direction dependent magnfication. This induces changes in amplitude, phase and direction (thus polarization).
\item[4.] Compute the electric fields present at the imaging plane from the angular spectrum in step 3.
\end{enumerate}

Finally the resulting image is computed with via the poynting vector integrated over pixels.

\begin{equation*}
  I = \left | E_{inc} + E_{s} \right |^2 
\end{equation*}

\section{Propagation of the Incident Field}
Git'er done boyo.

\section{Propagation of the Scattered Field}

Blah blah.

\section{Image formation}

Images brah.

\section{Comparison to Scalar Diffraction Theory}

More work.

\section{Discussion}

Not much to discuss yet.


