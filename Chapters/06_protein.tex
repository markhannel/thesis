\chapter{Holographic Characterization of Protein Aggregates}
\label{ch:protein}

\section{Introduction}
\label{sec:introduction}

This chapter demonstrate how holographic video microscopy can be used to detect, count and characterize individual micrometer-scale protein aggregates as they flow down a microfluidic channel in their native buffer. As has been shown in previous chapters, holographic characterization directly measures the radius and refractive index of subvisible protein aggregates, and offers insights into their morphologies. The measurement proceeds fast enough to build up population averages for time-resolved studies, and lends itself to tracking trends in protein aggregation arising from changing environmental factors. Information on individual particles’ refractive indexes can be used to differentiate protein aggregates from such contaminants as silicone droplets. These capabilities are demonstrated through measurements on samples of bovine pancreas insulin aggregated through centrifugation and of bovine serum albumin aggregated by complexation with a polyelectrolyte. Differentiation is demonstrated with samples that have been spiked with separately characterized silicone spheres. Holographic characterization measurements are compared with results obtained with micro-flow imaging and dynamic light scattering \cite{wang16}.

The tendency of proteins to aggregate into clusters
is a major concern in manufacturing protein-based
biopharmaceuticals and assessing their safety.
Beyond their reduced therapeutic efficacy, protein aggregates elicit immune responses that result in clinical adverse events \cite{FDA14, schellekens02, wang05, carpenter09, singh10, denengelsman11}.
Detecting, counting and characterizing protein aggregates
is essential to understanding the critical pathways responsible
for protein aggregation.
Established particle characterization technologies such as
dynamic light scattering (DLS) work well for \emph{in situ} characterization
of sub-micrometer-scale aggregates but less effective for larger subvisible aggregates \cite{panchal14, hamrang15}.
Others techniques, such as microflow imaging (MFI) \cite{sharma10, zolls13}, are better suited for aggregates larger than a micrometer or so \cite{hamrang15, zolls13}, but do not provide information on composition.
Comparatively few established techniques probe the size and composition of protein aggregates in the the subvisible range from 
\SI{100}{\nm} to \SI{10}{\um} \cite{hamrang15}.
The need for enhanced characterization techniques is particularly
acute in applications that require real-time monitoring of
subvisible aggregates in their native environment.

Here we show that holographic particle characterization is ideally sorted for detecting, counting
and characterizing individual colloidal particles in heterogeneous suspension. This technique naturally distinguishes protein aggregates from contaminants such as silicone droplets based on differences in their optical properties. Holographic characterization
rapidly builds up population statistics on subvisible protein aggregates
in their natural state, without dilution or special solvent conditions, and without the need for chemical or optical labels.
Its precision, accuracy and range of operation are demonstrated through measurements on model suspensions of colloidal spheres. Its utility for characterizing protein aggregates is illustrated through measurements on aggregates grown in solutions of bovine pancreas insulin (BPI) and bovine serum albumin (BSA), including measurements on samples deliberately spiked with silicone spheres. Comparisons with MFI and DLS highlight strengths and weaknesses of holographic characterization for analyzing subvisible protein aggregates in the micrometer size range. This study concludes by assessing how holographic characterization complements established particle-characterization techniques for analyzing aggregation in protein formulations.

\section{Characterization methods}
\label{sec:methods}

\subsection{Holographic characterization}
\label{sec:characterization}

\begin{figure}[!t]
  \centering
  \includegraphics[width=0.95\textwidth]{generalmethod4}
  \caption[Bovine insulin distribution]{Protein aggregates flowing down a microfluidic
    channel form holograms as they pass through a laser beam.
    A typical experimental hologram is reproduced as a grayscale image
    in the figure.
    Each hologram is recorded by a video camera and
    compared with predictions of Lorenz-Mie theory to
    measure each aggregate's radius, $a_p$, and refractive index, $n_p$.
    The scatter plot shows experimental data for \num{3000}
    subvisible aggregates of bovine insulin, with each data point
    representing the properties of a single aggregate, and colors
    denoting the relative probability density $\rho(a_p,n_p)$ for
    observations in the $(a_p,n_p)$ plane.}
  \label{fig:hologram}
\end{figure}

Our measurement technique, shown schematically in
Fig.~\ref{fig:hologram},
is based on in-line holographic
video microscopy \cite{sheng06,lee07,lee07a}, a technique
that creates holograms of individual objects in the microscope's
field of view.
A holographic microscope illuminates
its sample with a collimated laser beam rather than
a conventional incoherent light source.
Light scattered by an aggregate therefore
interferes with the remainder of the beam in the focal
plane of an optical microscope.  The microscope
magnifies this interference pattern and projects it onto the
face of a video camera, which records the spatially varying
intensity pattern,
$I(\vec{r})$.
Each image in the resulting video stream is a holographic
snapshot of the scatterers passing through the laser beam,
and can be analyzed with predictions 
\cite{lee07a,cheong10a,krishnatreya14} based on the
Lorenz-Mie theory of light scattering 
\cite{bohren83,mishchenko02,gouesbet11}
to measure each aggregate's radius, $a_p$, refractive index, $n_p$,
and three-dimensional position, $\vec{r}_p$ \cite{lee07a}.

Holographic characterization originally was developed for 
analyzing spherical particles \cite{lee07a,cheong09}.
Rigorously generalizing the analysis to account for the
detailed structure of aspherical and inhomogeneous objects 
is prohibitively slow because of the analytical complexity
and the associated computational
burden \cite{fung11,perry12,fung12}.
We therefore use the idealized spherical model
to characterize protein aggregates
with the understanding that the results
should be interpreted as referring to an
effective sphere comprising both the protein aggregate
and the fluid medium that fills out the effective sphere.
We previously have demonstrated that this approach
yields reliable results for porous colloidal
particles \cite{cheong11} and for dimpled spheres \cite{hannel15}.
The goal of this study is to demonstrate 
its utility for subvisible protein aggregates.

\subsubsection{Holographic video microscope}

Our custom-built holographic microscope uses a standard
microscope objective lens
(Nikon Plan Apo, $100\times$, numerical aperture 1.45,
oil immersion) in combination with a
tube lens to attain a system magnification of
\SI{135}{\nm\per pixel} on the face of a monochrome
video camera (NEC TI-324AII).
We illuminate the sample with the collimated beam
from a solid state laser (Coherent DPSS), which delivers
\SI{10}{\milli\watt} of light to the sample
vacuum wavelength of \SI{532}{\nm},
and a peak irradiance of \SI{10}{\milli\watt\per\square\mm}.
For micrometer-scale colloidal spheres, this instrument
can measure an individual particle's radius 
with nanometer precision \cite{moyses13,cheong09},
its refractive index to within a part per thousand
\cite{shpaisman12,moyses13}, and
can track its position to within a nanometer in the plane
and to within 3 nanometers along the optical axis
\cite{cheong10a,moyses13,krishnatreya14}.
Each fit can be performed in a few tens of milliseconds
using automated feature detection \cite{krishnatreya14a}
and image recognition algorithms \cite{yevick14}.
A single fit suffices to characterize a single protein aggregate.

\subsubsection{Holographic characterization procedure}

To characterize the population of aggregates
in a protein dispersion, we flow the sample through
the microscope's
observation volume in a microfluidic channel \cite{cheong09}.
Given the camera's exposure time of \SI{0.1}{\ms}, results
are immune to motion blurring for flow rates up to
\SI{100}{\um\per\second} \cite{cheong09,dixon11}.
Under typical conditions, no more than ten protein
aggregates pass through the \SI{86 x 65}{\um} field of
view at a time.
These conditions simplify the holographic analysis by minimizing overlap
between individual particles' scattering patterns.
Each aggregate typically is recorded in multiple video frames
as it moves through the field of view.
Such sequences of measurements are linked into trajectories
using a maximum-likelihood algorithm \cite{crocker96}
and median values are reported for each trajectory \cite{cheong09}.
These considerations establish an upper limit to the range
of accessible aggregate concentrations of \SI{e8}{aggregates\per\mL}.
At the other extreme, \SI{10}{\minute} of data suffices to
detect, count and characterize aggregates at concentrations as
low as \SI{e4}{aggregates\per\mL}. We pass samples through microfluidic channels with an optical path length of \SI{30}{\um} to ensure good imaging conditions for all aggregates, regardless of their height in the channel.

The scatter plot inset into Fig.~\ref{fig:hologram} shows typical results
for subvisible aggregates of bovine insulin.
Each point represents the properties of a single aggregate,
and is comparable in size to the estimated errors in the
radius and refractive index \cite{lee07a,krishnatreya14}.
Colors represent the local density $\rho(a_p,n_p)$ of recorded data
points in the $(a_p, n_p)$ plane, computed with a kernel density
estimator \cite{silverman92}, with red indicating the most
probable values.

Matching the refractive index of the medium to that of the protein suppresses light scattering by protein aggregates and thus reduces the contrast and the signal-to-noise ratio of recorded holograms. Holographic characterization is more effective for samples with larger index mismatch between the medium and the scattering particles. This limitation is shared by any non-fluorescent imaging technique.

\subsubsection{Holographic morphology measurements}

The same holograms used for holographic characterization through
Lorenz-Mie analysis also can be used to visualize the
three-dimensional morphology of individual aggregates through
Rayleigh-Sommerfeld back-propagation \cite{lee07} with 
volumetric deconvolution \cite{dixon11a}.
The complex scattered field in the focal plane could be reconstructed at 
height $z$ above the focal plane as the convolution
\begin{equation}
	E_R(\vec{r},z)=E_R(\vec{r},0) \otimes h(\vec{r},-z)
\end{equation}
of the scattered amplitude in the focal plane with the Rayleigh-Sommerfeld
propagator \cite{goodman05}
\begin{equation}
	h(\vec{r},-z)=\frac{1}{2\pi} \frac{\partial}{\partial z} \frac{e^{ikR}}{R},
\end{equation}
where $R^2 = r^2 + z^2$ and $k = 2 \pi n/\lambda$ is the light's wavenumber in a medium of refractive index $n$.
This technique uses the Rayleigh-Sommerfeld diffraction integral
to reconstruct the volumetric light field responsible for the
observed intensity distribution.
The object responsible for the scattering pattern appears in this
reconstruction in the form of the caustics it creates in the light
field \cite{lee07,cheong10a}.
For objects with features comparable in size to the wavelengths of
light, these caustics have been shown to accurately track the position
and orientation of those features in three dimensions \cite{cheong10}.
Deconvolving the resulting volumetric data set with the point-spread
function for the Rayleigh-Sommerfeld diffraction kernel
eliminates twin-image artifacts and yields a three-dimensional
representation of the scatterer \cite{dixon11a}.

Volumetric reconstructions of protein aggregates can be projected
into the imaging plane to obtain the equivalent of bright-field images
in the plane of best focus.  This reaps the benefit of holographic
microscopy's very large effective depth of focus compared with
conventional bright-field microscopy.  The resulting images are
useful for micro-flow imaging analysis, including analysis of
protein aggregates' sizes and morphology. An aggregate’s radius can be estimated from the projected reconstruction as the radius of gyration of the aggregate’s image.

In addition to providing an alternative approach to measuring aggregate size, this holographic approach to micro-flow imaging can be used to assess the rate of
false feature identifications in the Lorenz-Mie analysis, and thus
the rate at which large highly asymmetric aggregates are misidentified as clusters
of smaller aggregates.
Investigating aggregate morphology with holographic deconvolution
microscopy is a useful complement to holographic characterization through
Lorenz-Mie analysis. Whereas Lorenz-Mie fits proceed in a matter of milliseconds,
however, Rayleigh-Sommerfeld back-propagation is 
hundreds of times slower.  This study focuses, therefore,
on the information that can be obtained rapidly through
Lorenz-Mie characterization.  

\subsection{Dynamic light scattering}

Dynamic light scattering measurements were performed with a
Zetasizer Nano ZS (Malvern Instruments), which operates
in the back-scattering geometry at a vacuum wavelength
of \SI{633}{\nm}.
Samples for analysis were introduced into the disposable polystyrene
cuvette (DTS0012, Malvern Instruments) and allowed to thermally equilibrate
to \SI{25}{\celsius} within the instrument for \SI{5}{\minute}.
Each sample was scanned \num{10} times, and the average
results for particle counts were tabulated.

\section{Preparation of sample materials}
\label{sec:proteinpreparation}

\subsection{Preparation of bovine pancreas insulin aggregates}
\label{sec:biprep}

Samples of bovine pancreas insulin 
(Mw: \SI{5733.49}{\dalton}, Sigma-Aldrich, CAS number: 11070-73-8) 
were prepared according to previously published methods
\cite{sluzky91,costantino94} 
with modifications for investigating insulin aggregation induced by
agitation alone.
Insulin was dissolved at a concentration of \SI{5}{\mg\per\mL}
in \SI{10}{\milli M} Tris buffer (Life Technologies, CAS
number 77-86-1). The pH of the buffer was adjusted to 7.4 with \SI{37}{\percent}
hydrochloric acid (Sigma Aldrich, CAS number: 7647-01-0).
The solution then was centrifuged at \SI{250}{rpm} for \SI{1}{\hour}
to induce aggregation, at which time the sample still appeared transparent to visual inspection.

\subsection{Preparation of bovine serum albumin aggregates}
\label{sec:bsaprep}

Solutions of bovine serum albumin
(BSA) (Mw: \SI{66500}{\dalton}, Sigma Aldrich, CAS number: 9048-46-8)
were aggregated by complexation with poly(allylamine hydrochloride)
(PAH) (Mw: \SI{17500}{\gram\per\mole}, CAS number: 71550-12-4, average
degree of polymerization: \num{1207}) \cite{ball02,hagiwara96}.
BSA and PAH were dissolved in \SI{10}{\milli M} Tris-HCl (pH 7.4) buffer
(Life Technologies, CAS number: 77-86-1) to achieve concentrations
of \SI{1.22}{\mg\per\mL} and \SI{0.03}{\mg\per\mL}, respectively.
The reagents were mixed by vortexing to ensure dissolution, and
aggregates formed after the sample was allowed to equilibrate for one
hour.

Additional samples were prepared under comparable conditions
with the addition of \SI{0.1}{M} NaCl (Sigma Aldrich, CAS number
7647-14-5) to facilitate complexation and thus to promote aggregation.

Control samples were prepared without the addition of salt or PAH, and were measured immediately after preparation.

\subsection{Preparation of a stoichiometric mixture of colloidal spheres}
\label{sec:mixture}

The standard sample is constituted as a mixture of four
populations of monodisperse colloidal spheres in which
each population has a distinct mean size and composition.
The monodisperse spheres were purchased from Bangs Laboratories
as aqueous dispersions at \SI{10}{\percent} solids.
Stock suspensions were diluted one-thousand-fold with deionized
water and then were combined in equal volumes to create a
heterogeneous mixture.
The four populations in this mixture are
polystyrene spheres of diameter $2a_p = \SI{0.71(9)}{\um}$
(Catalog Code PS03N, Lot Number 9402)
and $2a_p = \SI{1.58(14)}{\um}$
(Catalog Code PS04N, Lot Number 9258),
and silica spheres of diameter $2a_p = \SI{0.69(7)}{\um}$
(Catalog Code SS03N, Lot Number 8933)
and $2a_p = \SI{1.54(16)}{\um}$
(Catalog Code SS04N, Lot Number 5305).
The quoted range of particle sizes is estimated by
the manufacturer using
dynamic light scattering for the smaller spheres,
and by the Coulter principle for the larger spheres.

Additional samples of colloidal polystyrene spheres dispersed in water were used to establish the size range for holographic characterization and were not included in the stoichiometric mixture. These samples have nominal radii of \SI{0.20(5)}{\um} (Catalog Code PS02N, Lot Number 6379), \SI{0.78(9)}{\um} (Catalog Code PS04N, Lot Number 9258) and \SI{10.6(7)}{\um} (Catalog Code PS07N, Lot Number 11218).

\subsection{Preparation of silicone spheres}
\label{sec:silicone}

Silicone spheres composed of polydimethylsiloxane (PDMS)
were synthesized by base catalyzed hydrolysis and copolymerization
of difunctional diethoxydimethylsilane (DEDMS) 
(Sigma-Aldrich, CAS number 78-62-6, \SI{3}{vol\percent})
and trifunctional triethoxymethylsilane (TEMS) 
(Sigma-Aldrich, CAS number 2031-67-6, \SI{2}{vol\percent})
following a standard protocol \cite{obey94,goller97}.
A mixture of DEDMS and TEMS with 60:40 stoichiometry
is added into deionized water
(Millipore MilliQ, \SI{93}{vol\percent}) at \SIrange{28}{30}{wt\percent}
and ammonium hydroxide solution (ACROS Organics \SI{2}{vol\percent})
to obtain a total volume of \SI{10}{\mL}.
The sample was shaken vigorously with a vortexer for \SI{4}{min}
at room temperature to initiate nucleation, and then
left to polymerize on a rotating frame at \SI{10}{rpm} for three hours.
Fully grown silicone spheres were then mixed with
suspensions of protein aggregates at a volume fraction of \num{e-5}
to obtain an effective concentration of spheres of
\SI{4e6}{\per\mL}.  

The polydispersity in radius of these particles can be varied by changing the duration of the mixing process \cite{obey94,goller97, wang15}. For the present study, we synthesized particles with mean radii around \SI{1.5}{\um} in one batch with 12\% polydispersity and another with 32\% polydispersity. These different size distributions are intended to have distinctive signatures for holographic characterization.

Lightly polymerized silicone spheres share most properties with unpolymerized silicone oil droplets. Their mean refractive index, \num{1.388(2)}, only slightly exceeds that of DEDMS, \num{1.381}, and TEMS, \num{1.383}, as determined with an Abbe refractometer (Edmund Optics) and by holographic characterization \cite{lee07a, shpaisman12, wang15a}.
The stock suspension of silicone spheres was mixed with
suspensions of protein aggregates at \SI{10}{vol\percent}
to obtain an effective concentration of spheres of
\SI{4e6}{\per\mL}. 

%\section{Results}
%\label{sec:results}

\section{Verification of precision and accuracy}
\label{sec:verification}

\begin{figure}[!t]
  \centering
  \includegraphics[width=0.95\columnwidth]{quad3}
  \caption[Holographic characterization of four types of particles]
    {(a) Scatter plot of radius $a_p$ and refractive index $n_p$
    obtained with holographic characterization of
    the four-component stoichiometric colloidal mixture described in the Materials 
    section. Results for \num{20000} particles
    are plotted.  Superimposed crosses indicate the
    manufacturer's specification for each of the four populations.
    These results establish holographic characterization's ability
    to differentiate particles by composition as well as by size.
    (b) Measured holograms of colloidal polystyrene spheres in water together with 
    fits, demonstrating the range of particle sizes amenable to holographic
    characterization. These typical examples were obtained for spheres with 
    radii $a_p = \SI{0.237}{\um}$ (\SI{224 x 224}{pixel} region of interest),
    \SI{0.800}{\um} (\SI{356 x 356}{pixel}) and \SI{10.47}{\um} 
    (\SI{608 x 608}{pixel}). The fit to each hologram yields values 
    for the particle’s radius, $a_p$, and refractive index, $n_p$. Radial 
    profiles, $b(r)$, are obtained from these holograms and their fits by 
    averaging the normalized intensity over angles around the center of each 
    feature, and are plotted as a function of distance $r$ from the center 
    of the feature. Experimental data are plotted as blue curves within shaded 
    regions representing the measurement’s uncertainty at that radius. Fits are 
    superimposed as orange curves and closely track the experimental data.}
  \label{fig:quad}
\end{figure}

Fig.~\ref{fig:quad}(a) demonstrates the unique and distinguishing ability of holographic characterization to correctly identify distinct populations of colloidal particles in heterogeneous suspensions. 
These data were obtained through holographic
characterization of a model colloidal dispersion
consisting of a stoichiometric mixture of
four distinct types of monodisperse colloidal spheres.
Each of the four peaks in Fig.~\ref{fig:quad}(a) represents the
properties of one of those populations.
  
In each case, the holographically
measured distribution of properties is consistent
with the manufacturer's specification.
This agreement, together with complementary
tests in previous publications 
\cite{lee07a,cheong11,shpaisman12,moyses13,krishnatreya14},
establishes the precision and accuracy of particle-resolved
holographic characterization.
Specifically, these results demonstrate that this implementation of holographic characterization can yield the radius of an individual micrometer-scale sphere with a precision of \SI{10}{\nm} and an accuracy of \SI{50}{\nm} \cite{krishnatreya14}, and the refractive index to within 5 parts per thousand \cite{shpaisman12}. Previous reports \cite{cheong11, hannel15} demonstrate comparably good results for porous spheres and otherwise irregular particles.

Fig.~\ref{fig:quad}(b) offers an experimental demonstration of the range of particle sizes over which holographic characterization yields useful results. The three holograms presented here were recorded for three different polystyrene spheres dispersed in water, one with a radius of $a_p = \SI{0.237}{\um}$, at the small end of the technique’s effective range, one with a radius of $a_p = \SI{0.800}{\um}$, and the third with a radius of $a_p = \SI{10.47}{\um}$. These measured holograms are presented alongside corresponding pixel-by-pixel fits to the predictions of the theory of light scattering, which are parameterized by each particle’s three-dimensional position, radius and refractive index. The quality of a fit can be assessed by plotting the radial profile of the normalized image intensity, $b(r)$. This is obtained by averaging the two-dimensional intensity pattern over angles around the center of the scattering pattern. Curves obtained from the measured data are plotted in Fig.~\ref{fig:quad}(b) within shaded regions that represent measurement uncertainties. Curves obtained from the fits are overlaid on the experimental data for comparison. The fits track the experimental data extremely well over the entire range of particle sizes.
This data set illustrates the unique ability of holographic
characterization to characterize heterogeneous colloidal dispersions.

\section{Holographic characterization of subvisible protein aggregates}

\subsection{Subvisible insulin aggregates}
\label{sec:insulin}

Although the bovine insulin sample appeared clear under visual inspection,
the data in Fig.~\ref{fig:hologram} reveal
a concentration of \num{3.9(5)e7} subvisible aggregates per milliliter,
which corresponds to a volume fraction of roughly \num{e-3}.
Uncertainty in this value results from feature identification errors
for the largest particles and uncertainty in the flow speed.
Aggregates with radii smaller than \SI{200}{\nm} are not detected
by the holographic characterization system 
and therefore were not counted in these totals.
The distribution of particle characteristics is peaked at a radius of
\SI{1.6}{\um}, and is both broad and multimodal.
No aggregates were recorded with radii exceeding \SI{4.2}{\um},
which suggests that such large-scale aggregates are present
at concentrations below \SI{e4}{\per\mL}.

The aggregates' refractive indexes vary over a wide range
from just above that of the buffer, $n_m = \num{1.335}$, 
to slightly more than \num{1.42}.
This range is significantly smaller
than the value around \num{1.54} that would be expected 
for fully dense protein crystals \cite{maschke06}.  
This observed upper limit is consistent, however, 
with recent index-matching
measurements of the refractive index of protein aggregates
\cite{zolls13}.
These latter measurements were performed by perfusing
protein aggregates with index-matching fluid,
and therefore yield an estimate for the refractive index
of the protein itself.
Holographic characterization, by contrast, analyzes an
effective scatterer comprised of both the higher-index
protein and also the lower-index buffer that fills out the sphere.
We previously have shown that such an effective sphere
has an effective refractive index intermediate between that
of the two media \cite{cheong11} in a ratio that depends
on the actual particle's porosity.  More porous or open structures
have smaller effective refractive indexes.
The influence of porosity on the effective refractive index,
furthermore, is found to be proportionally larger for particles 
with larger radii \cite{cheong11}.

The effective sphere model account for general
trends in the holographic characterization data under
the assumption that the protein aggregates have 
open irregular structures.
This proposal is consistent with previous \emph{ex situ}
studies that have demonstrated that bovine insulin forms filamentary aggregates
\cite{omichi14,yip98,yip98a}.

The particular ability of holographic characterization to record
both the size and the refractive index of individual colloidal
particles therefore offers insights into
protein aggregates' morphology \emph{in situ} and
without dilution and without any other special preparation.
This capability also enables holographic characterization to
distinguish protein aggregates from common contaminants
such as silicone oil droplets and rubber particles 
\cite{wang15}, which pose problems for other analytical techniques
\cite{demeule10}.

\subsection{Subvisible BSA-PAH complexes}
\label{sec:bsa}

\begin{figure}[!t]
  \centering
  \includegraphics[width=0.9\textwidth]{bsa_runs10}
  \caption[Influence of salt on BSA-PAH complexes]
    {Influence of added salt on the measured distribution 
    of the radius $a_p$ and refractive index $n_p$
    of BSA-PAH complexes.
    Each point in the scatter plots represents the properties of a single aggregate
    and is colored by the relative density of observations,
    $\rho(a_p,n_p)$.  Upper panels present the associated
    size distribution $\rho(a_p)$ within a shaded region
    representing the instrumental and statistical error.
    (a) BSA complexed with PAH in Tris buffer (1100 aggregates).
    (b) The same sample with \SI{0.1}{M} NaCl (1200 aggregates).
    (c) and (d) present the projected relative size distributions, $\rho(a_p)$, 
    from (a) and (b), respectively. Shaded regions represent the instrumental and
    statistical error.}
  \label{fig:bsa_runs}
\end{figure}

Figure~\ref{fig:bsa_runs} shows comparable holographic
characterization results for the two samples of bovine serum albumin complexed with PAH under differing salt concentrations.
The data in Fig.~\ref{fig:bsa_runs}(a) were obtained for the sample
prepared without additional salt.
As for the insulin sample, holographic characterization of the BSA
sample reveals 
\SI{9.8(5)e6}{aggregates\per\mL} in the range of radii running
from \SI{300}{\nm} to \SI{2.5}{\um}, and a
peak radius of \SI{0.5}{\um}.
Although holographic characterization is capable of detecting aggregates with 
radii up to \SI{10}{\um}, no aggregates were observed with radii exceeding \SI{2.5}{\um}.
This suggests that larger aggregates are present at concentrations
below \SI{e4}{\per\mL}. The plot range is selected accordingly.

The trace in Fig.~\ref{fig:bsa_runs}(c) is a projection of the
joint distribution, $\rho(a_p,n_p)$, into the distribution of
aggregate sizes, $\rho(a_p)$, obtained by integrating over $n_p$. 
This projection more closely resembles results
provided by other size-measurement techniques, such as micro-flow imaging 
and dynamic light scattering, that report size data, but no other information.

As for the BI samples, the anticorrelation between $a_p$ and $n_p$
evident in \ref{fig:bsa_runs} suggests that BSA-PAH complexes have
an open structure.  This is consistent with previous \emph{ex situ}
studies \cite{hagiwara96,siposova12} that demonstrate
that BSA aggregates into weakly branched structures.
Adding salt is known to enhance complexation \cite{ball02}
in the BSA-PAH system \cite{ball02}.
Comparing Fig.~\ref{fig:bsa_runs}(b) with Fig.~\ref{fig:bsa_runs}(a) 
suggests that adding salt  increases the mean aggregate size by nearly a factor of two,
and also substantially broadens the distribution of aggregate sizes,
These trends can be seen in corresponding projections in 
Fig.~\ref{fig:bsa_runs}(c) and Fig.~\ref{fig:bsa_runs}(d).
What the projected size distributions fail to capture is the striking change in the
joint distribution of aggregate radii and refractive indexes
from Fig.~\ref{fig:bsa_runs}(a) to Fig.~\ref{fig:bsa_runs}(b).
This shift suggests that the larger aggregates grown in the
presence of added salt are substantially more porous
\cite{cheong11}.  This insight into the aggregates'
morphology would not be offered by the
size distributions alone.

The number of aggregates detected in Fig.~\ref{fig:bsa_runs}(b) does not differ significantly from the number observed in Fig.~\ref{fig:bsa_runs}(a). Adding salt therefore appears to increase the size of micrometer-scale subvisible aggregates without appreciably increasing their concentration.

As a control, we performed a holographic characterization measurement on a BSA solution with no added PAH or salt, and no deliberate agitation. A \SI{10}{\minute} data set reveals a total of 98 features, corresponding to a concentration of \SI{9(3)e5}{aggregates\per\mL} in the accessible size range. This is an order of magnitude fewer aggregates than is observed in the presence of the complexing agent.

\section{Comparison with established techniques}

\subsection{Microflow imaging}
\label{sec:morphology}

\begin{figure}[!t]
  \centering
  \includegraphics[width=1.0\textwidth]{catalog7a}
\end{figure}
%\addtocounter{figure}{-1}
\begin{figure}[!t]
	\caption[Influence of aggregate morphology on holographic
    characterization]
    {Influence of aggregate morphology on holographic
    characterization.  Holograms of typical aggregates arranged in
    order of increasing discrepancy between measured and fit
    holograms.  Column (a) shows \SI{160 x 160}{pixel}
    regions of interest from the microscope's field of view, centered
    on features automatically identified as candidate BSA-PAH complexes.
    Column (b) shows fits to the Lorenz-Mie theory for
    holograms formed by spheres.  Column (c) shows radial
    profiles of the experimental hologram (black curves) overlaid with the
    radial profile of the fits (red curves).  Shaded regions represent
    the estimated experimental uncertainty.
    Column (d) shows Rayleigh-Sommerfeld reconstructions of the aggregates'
    three-dimensional structures obtained from the experimental
    holograms.  Grayscale images are projections of the
    reconstructions, which resemble equivalent bright-field images at
    optimal focus.  Superimposed circles indicate fit estimates for
    the particle size.}
	\label{fig:morphology}
\end{figure}

Fig.~\ref{fig:morphology} presents direct comparisons between holographic 
characterization and micro-flow imaging (MFI) for a representative sample 
of six BSA-PAH complexes whose morphologies range from nearly spherical 
compact clusters to extended spindly structures.
Each aggregate's hologram is compared with a
nonlinear least-squares fit to the predictions of Lorenz-Mie theory.  
The reduced $\chi^2$ metric for these fits is used to arrange the results from 
best fits at the top to worst fits at the bottom.
The reduced $\chi^2$ is defined as \cite{taylor97}
\begin{equation}
\chi^2=\frac{1}{n-c} \sum_{pixel}^{n} \left( \frac{\textnormal{measurement -
      theoretical}}{\textnormal{single\ pixel\ errors}} \right)^2,
\end{equation}
where $n$ is the number of pixels for each experimental hologram, $c$ is the number of 
fitting parameters and $d-c$ is called the number of degrees of freedom. The light 
intensity difference between the experimental image and the theoretical calculation at 
each pixel is normalized by the noise at the corresponding pixel. The noise level is 
determined by the temporal and spatial variations of pixel values recorded from the 
camera.
Each of the measured holograms also is used to reconstruct a
volumetric image of the individual aggregate through
Rayleigh-Sommerfeld deconvolution microscopy \cite{dixon11a}. The size of the reconstructed cluster then can be compared with the effective radius obtained from holographic characterization.

Even the two most compact clusters appear to be substantially aspherical.
Their holograms, nevertheless, are very well reproduced by the nonlinear fits.
The reduced $\chi^2$ metrics for these fits are close to unity, suggesting
that the model adequately describes the light-scattering process
and that the single-pixel noise is well estimated.
Values for the aggregate radius are consistent with the size estimated
from Rayleigh-Sommerfeld reconstruction.
Circles with the holographically-determined radii are superimposed on the 
numerically refocussed bright-field images in Figure~\ref{fig:morphology}(d) 
for comparison.
This success is consistent with previous comparisons of Lorenz-Mie and
Rayleigh-Sommerfeld analyses for colloidal spheres \cite{cheong10a}
and colloidal rods \cite{cheong10}.

Errors increase as aggregates become increasingly highly structured
and asymmetric.  Even so, estimates for the characteristic size
are in reasonable agreement with the apparent size of the bright-field
reconstructions even for the worst case.
This robustness arises because the effective size of the scatterer
strongly influences the size and contrast of the central scattering
peak and the immediately surrounding intensity minimum.
Faithful fits in this region of the interference pattern therefore yield
reasonable values for the scatterer's size.
The overall contrast of the pattern as a whole encodes the scatterer's
refractive index, and thus is very low for such open-structured
clusters.

These representative examples are consistent with
earlier demonstrations \cite{cheong11,hannel15} that holographic
characterization yields useful characterization data for
imperfect spheres and aspherical particles.
Particularly for larger aggregates, the estimated value
for the refractive index describes an effective sphere.
The estimated radius, however, is a reasonably robust
metric for the aggregate's size.

Independent of the ability of holographic characterization to provide
insight into morphology, these results demonstrate
that holographic microscopy usefully detects and counts
subvisible protein aggregates in solution.
These detections by themselves
provide useful information for characterizing the state of aggregation
of the protein solution \emph{in situ} without requiring
extensive sample preparation.
Holographic microscopy's large effective depth of field
then serves to increase the analysis rate relative to
conventional particle imaging analysis \cite{filipe10}.

\begin{figure}[!t]
  \centering
  \includegraphics[width=0.75\textwidth]{mfi7}
	\caption[Microflow imaging results]
    {Comparison of size distributions measured with microflow imaging and 
    holographic characterization. Each bin represents the number of particles 
    per milliliter of solution in a range of $\pm \SI{100}{\nm}$ about the bin’s central 
    radius. (a) Sample without added salt from Fig.~\ref{fig:bsa_runs}(a). (b) Same with 
    added NaCl from Fig.~\ref{fig:bsa_runs}(b).}
	\label{fig:mfi}
\end{figure}

Like holographic characterization, MFI yields particle-resolved radius measurements 
that can be used to calculate the concentration of particles in specified size bins.
These results may be compared directly with projected size distributions produced by 
holographic characterization. The data in Fig.~\ref{fig:mfi} show such a comparison for 
the BSA-PAH complexes with and without added salt featured in Fig.~\ref{fig:bsa_runs}.
Results are presented as the number, $N(a_p)$, of aggregates per milliliter in a size 
range of $\pm \SI{100}{\nm}$ around the center of each bin in $a_p$ . The holographic 
characterization data from Fig.~\ref{fig:bsa_runs}(b) are rescaled in this plot for 
comparison. Independent studies demonstrate that MFI analysis yields reliable size 
estimates for aggregates with radii larger than \SI{1}{\um} \cite{weinbuch13, zolls13}. 
Diffraction causes substantial measurement errors for smaller particles. We therefore 
collect MFI results for smaller particles into \SI{600}{\nm}-wide bins in Fig.~
\ref{fig:mfi}(a) and Fig.~\ref{fig:mfi}(b), with corresponding normalization. The lower 
end of this bin’s range corresponds with the smallest radii reported by holographic 
characterization. Agreement between holographic characterization and MFI is reasonably 
good over the entire range of particle sizes plotted. Both techniques yield consistent 
values for the overall concentration of \SI{107}{aggregates \per \mL}. MFI systematically 
reports larger numbers of aggregates on the large end of the size range and fewer on the 
small end. This difference can be attributed to the most elongated and irregular 
aggregates, such as the last example in Fig.~\ref{fig:morphology}, whose size is 
underestimated by holographic characterization. This effect of morphology on holographic 
size characterization has been discussed previously \cite{hannel15}. Even in these cases, 
holographic characterization correctly detects the particles’ presence and identifies 
them as micrometer-scale objects. Both MFI and holographic characterization yield 
consistent results for the total number density of aggregates.

For particles on the smaller end of the size range, MFI provides particle counts, but no 
useful characterization data. Holographic characterization, by contrast offers reliable 
size estimates in this regime. Over the entire range of sizes considered, holographic 
characterization also provides estimates for particles’ refractive indexes.

\subsection{Dynamic light scattering}

\begin{figure}[!t]
  \centering
  \includegraphics[width=0.75\textwidth]{zetasizer3}
	\caption[Dynamic light scattering results]
    {Characterization of BSA-PAH complexes by dynamic light scattering (DLS). Values 
    represent the percentage, $P(a_h)$, of the scattered light’s intensity due to 
    scatterers of a given hydrodynamic radius, $a_h$. The arrow indicates a small peak in 
    both distributions around $a_h = \SI{2.8}{\um}$.}
	\label{fig:zetasizer}
\end{figure}

To verify the presence of subvisible protein aggregates in our samples, we also performed 
dynamic light scattering measurements. Whereas holographic characterization and MFI yield 
particle-resolved measurements, DLS is a bulk characterization technique. Values reported 
by DLS reflect the percentage, $P(a_h)$, of scattered light that may be attributed to 
objects of a given hydrodynamic radius, $a_h$. The resulting size distribution therefore 
is weighted by the objects’ light-scattering characteristics. Scattering intensities can 
be translated at least approximately into particle concentrations if the particles are 
smaller than the wavelength of light and if they all have the same refractive index 
\cite{berne00}. Direct comparisons are not possible when particles’ refractive indexes 
vary with size, as is the case for protein aggregates. In such cases, DLS is useful for 
confirming the presence of scatterers within a range of sizes. Fig.~\ref{fig:zetasizer} 
presents DLS data for the same samples of BSA-PAH complexes presented in
Fig.~\ref{fig:bsa_runs}. For both samples, DLS reveals the presence of an abundance of 
scatterers with radii smaller than \SI{100}{\nm}. The detection threshold of DLS for 
scatterers of this size is roughly \SI{108}{aggregates \per \mL}, as determined by 
independent studies \cite{panchal14}. We conclude that both samples have at least this 
concentration of submicrometer-diameter aggregates. Such objects are smaller than the 
detection limit for our implementation of holographic video microscopy, and so were not 
resolved in Fig.~\ref{fig:bsa_runs}.

The distribution shifts to larger sizes in the sample with added salt, consistent with 
the results of holographic characterization. Both samples show a very small signal, 
indicated by an arrow in Fig.~\ref{fig:zetasizer}, for subvisible objects whose 
hydrodynamic radius is \SI{2.8}{\um}. This confirms the presence of such scatterers in 
our sample at a concentration just barely above the detection threshold of DLS for 
objects of that size.

The sample with added salt also has a clear peak around $a_h = \SI{400(20)}{\nm}$ that is 
in the detection range of holographic characterization. The corresponding peak in 
Fig.~\ref{fig:bsa_runs}(b) appears at a substantially larger radius, 
$p = \SI{770(20)}{\nm}$. One likely source of this discrepancy is that holographic 
characterization reports the radius of an effective bounding sphere, whereas DLS reports 
the hydrodynamic radius, which can be substantially smaller for open structures. Another 
contributing factor is that larger aggregates have lower effective refractive indexes and 
thus scatter light proportionately less strongly than smaller aggregates 
\cite{feder84, cheong11}. This effect also shifts the apparent size distribution downward 
in DLS measurements. It does not, however, affect holographic characterization, which 
reports both the size and refractive index of each object independently.

\section{Holographic differentiation of silicone spheres from protein aggregates}

DLS cannot distinguish protein aggregates from other populations of particles in 
suspension. MFI can differentiate some such contaminants by morphology: silicone 
droplets, for example, tend to be spherical, whereas protein aggregates tend to have 
irregular shapes. Morphological differentiation works best for particles that are 
substantially larger than the wavelength of light, whose structural features are not 
obscured by diffraction. Through the information it provides on individual particles’ 
refractive indexes, holographic characterization offers an additional avenue for 
distinguishing micrometer-scale objects by composition. We demonstrate this capability by 
performing holographic characterization measurements on BSA samples that are deliberately 
adulterated with silicone spheres.

\subsection{Holographic characterization of silicone spheres}

\begin{figure}[!t]
  \centering
  \includegraphics[width=0.95\textwidth]{silicone2}
	\caption[Holographic characterization for silicone spheres]
    {Holographic characterization data for silicone spheres dispersed in deionized water. 
    The gray-shaded region denotes range of refractive indexes expected for these 
    particles based on their composition. (a) Monodisperse sample (600 spheres). (b) 
    Polydisperse sample (600 spheres).}
	\label{fig:silicone}
\end{figure}

Fig.~\ref{fig:silicone} shows holographic characterization data for silicone spheres 
dispersed in deionized water. The sample in Fig.~\ref{fig:silicone}(a) is comparatively 
monodisperse with a sample-averaged radius of \SI{0.75(9)}{\um}. The particles in 
Fig.~\ref{fig:silicone}(b) are drawn from a broader distribution of sizes, with a mean 
radius of \SI{0.87(28)}{\um}. Both samples of spheres have refractive indexes consistent 
with previously reported values for PDMS with 40\% crosslinking, $n_p = 1.388 \pm 0.005$ 
\cite{wang15}. This range is indicated with a shaded region in Fig.~\ref{fig:silicone}.

Unlike the protein aggregates, these particles’ refractive indexes are uncorrelated with 
their sizes. This is most easily seen in the polydisperse sample in 
Fig.~\ref{fig:silicone}(b), and is consistent with the droplets having uniform density 
and no porosity \cite{wang15}. We expect to see the same distribution of single-particle 
properties when these silicone spheres are co-dispersed with protein aggregates.

\subsection{Differential detection of silicone spheres}

\begin{figure}[!t]
  \centering
  \includegraphics[width=0.95\textwidth]{bsasilicone4}
	\caption[Holographic measurement for BSA-PAH complexes with added silicone spheres]
    {Holographic measurement of the relative probability density, $\rho(a_p,n_p)$, of 
    particle radius and refractive index for suspensions BSA-PAH complexes spiked with 
    added silicone spheres. (a) Sample prepared under the same conditions as in 
    Fig.~\ref{fig:bsa_runs}(a) spiked with the monodisperse spheres from 
    Fig.~\ref{fig:silicone}(a) (2000 features). (b) Sample prepared under the same 
    conditions as in Fig.~\ref{fig:bsa_runs}(b) spiked with the polydisperse spheres from 
    Fig.~\ref{fig:silicone}(b) (1600 features). (c) and (d) show the projected relative 
    probability density, $\rho(a_p)$, for particle radius from the data in (a) and (b), 
    respectively.}
	\label{fig:bsasilicone}
\end{figure}

The data in Fig.~\ref{fig:bsasilicone}(a) were obtained from a
sample of BSA prepared under the same conditions as
Fig.~\ref{fig:bsa_runs}(a), but with the addition of monodisperse
silicone spheres at a concentration of \SI{4e6}{particles\per\mL}.
The resulting distribution of particle properties is clearly bimodal
with one population resembling that obtained from 
protein aggregates alone, and the other having a refractive index
consistent with that of the silicone spheres, $n_p = \num{1.388(5)}$
\cite{wang15a}. The sample in Fig.~\ref{fig:bsasilicone}(b) similarly were prepared under conditions comparable to those from Fig.~\ref{fig:bsa_runs}(b) with the addition of polydisperse silicone spheres from the sample in Fig.~\ref{fig:silicone}(b). 
Consistency between features associated with protein
aggregates in Figs.~\ref{fig:bsa_runs} and Fig.~\ref{fig:bsasilicone} demonstrate
that both the sample preparation protocol and holographic
characterization yields reproducible results from sample to sample,
and that holographic characterization of protein aggregates is not
influenced by the presence of extraneous impurity particles.

Interestingly, both distributions feature a small peak around $a_p = \SI{2.8}{\um}$ that corresponds to the peak in the DLS data from Figs.~\ref{fig:zetasizer}. This feature is not present in Figs.~\ref{fig:bsa_runs}. It is likely that this very small population of larger aggregates was present in those samples, but at a concentration  below the threshold for detection in a \SI{10}{\minute} measurement.

The distributions of features associated with silicone droplets in 
Fig.~\ref{fig:bsasilicone} also agree well with the holographic characterization data on the droplets alone from Fig.~\ref{fig:silicone}.
These results demonstrate that the refractive-index data from
holographic characterization can be used to distinguish protein
aggregates from silicone oil droplets.
Such differentiation would not be possible on the basis of the size
distribution alone, as can be seen from the projected data in
Fig.~\ref{fig:bsasilicone}.

Holographic characterization cannot differentiate silicone droplets from protein clusters whose refractive index is the same as silicone’s. Such ambiguity arises for the smallest particles analyzed in Fig.~\ref{fig:bsasilicone}. Spherical silicone droplets sometimes can be distinguished from irregularly shaped protein aggregates under these conditions using morphological data obtained through deconvolution analysis of the same holograms. The distinction in these cases still would be less clear than can be obtained with Resonant Mass Measurement (RMM), which differentiates silicone from protein by the sign of their relative buoyancies \cite{weinbuch13}. In cases where specific particles cannot be differentiated unambiguously, the presence of silicone droplets still can be inferred from holographic characterization data because such particles create a cluster in the 
$(a_p,n_p)$ plane whose refractive index is independent of size. The relative abundances of the two populations then can be inferred, for example, by statistical clustering methods.

\section{Summary of key points}
\label{sec:summary}

The measurements presented here demonstrate that holographic video
microscopy together with Lorenz-Mie analysis can detect,
count and characterize subvisible protein aggregates.
Data acquisition is rapid, typically taking no more than
\SI{15}{\minute}, and requires no special sample preparation.
Our implementation is effective for aggregates ranging in radius
from \SI{300}{\nm} to \SI{10}{\um} and at concentrations from
\SI{e4}{aggregates\per\mL} to \SI{e8}{aggregates\per\mL}.

Large irregularly shaped protein aggregates can be mistaken for smaller aggregates by our present implementation of holographic characterization. Such cases are readily detected through the quality of fit and can be remedied by micro-flow imaging analysis of the same holograms. This additional analysis step also provides information on the morphology of individual protein aggregates.
The same holograms used for characterization measurements
also can be interpreted to estimate the morphology of individual
protein aggregates through numerical back-propagation.

Instrumentation for holographic characterization is compact and closely resembles the optical train for a conventional light microscope. Calibration requires values for the laser wavelength, the microscope’s magnification and the medium’s refractive index. The measurement process requires less than one hundred microliters of sample. Data recording and analysis is automated.

We anticipate, therefore, that holographic characterization of protein aggregates will be useful for assessing the stability of biopharmaceutical formulations throughout the product lifecycle. Access to time-resolved characterization data for micrometer-scale subvisible protein aggregates should prove useful for assessing the efficacy of stabilizing agents during product development. Automated continuous operation lends itself to process control for manufacturing. Variants of holographic characterization might even lend themselves to assessing the quality of biopharmaceutical products \emph{in situ}.

