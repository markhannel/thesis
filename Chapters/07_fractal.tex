\chapter{Holographic Characterization of Colloidal Fractal Aggregates}
\label{ch:fractal}

\section{Introduction}

In-line holographic microscopy images of micrometer-scale fractal
aggregates can be interpreted with an effective-sphere model
to obtain each aggregate's size and the population-averaged fractal dimension.
We demonstrate this technique experimentally using
model fractal clusters of polystyrene nanoparticles and 
fractal protein aggregates composed of bovine serum albumin and
bovine pancreas insulin \cite{wang16a}.

Holograms of micrometer-scale colloidal spheres
obtained with in-line holographic video microscopy
can be analyzed with the Lorenz-Mie theory of light
scattering to obtain each sphere's radius and refractive index,
typically with part-per-thousand precision \cite{lee07a,krishnatreya14}.
Characterizing a single sphere requires a few milliseconds
on a standard computer \cite{yevick14}, which is fast enough that several
thousand spheres can be characterized in under ten minutes.
Analyzing holograms of aspherical particles and colloidal
clusters is substantially more challenging \cite{fung12},
particularly if no information is available \emph{a priori}
about the particles' geometry.
We previously have demonstrated that the Lorenz-Mie analysis
developed for homogeneous spheres also yields useful
characterization data for porous spheres \cite{cheong11,wang15a}, 
dimpled spheres \cite{hannel15}, and protein aggregates \cite{wang16}.
The last of these applications treats each irregularly-shaped protein
aggregate as an effective sphere composed of the aggregate
itself and the fluid medium filling its pores.
Here, we develop an effective-sphere formalism for
Lorenz-Mie microscopy of fractal aggregates and
demonstrate its efficacy through measurements on model
systems.

Detecting and characterizing micrometer-scale aggregates
is useful both for fundamental research and also for solving 
real-world problems.
Protein aggregation, for example, is a critical concern
for the biopharmaceutical industry because it limits
the efficacy of protein-based drugs and can induce harmful
immunogenic responses in patients \cite{wang05}.
Information on the concentration, size distribution and morphology of protein
aggregates provides guidance for formulating stable products and for
avoiding adverse clinical outcomes.
Conventional light-scattering techniques do not work well
for particles in the relevant size range \cite{panchal14}, and cannot
distinguish aggregates of interest from other contaminants
commonly found in commercial formulations.
Similar detection and characterization challenges arise in
the precision slurries used by the semiconductor manufacturing
for chemical-mechanical planarization \cite{basim02}.
As a particle-resolved measurement technique, holographic
characterization naturally differentiates micrometer-scale
particles by size and composition \cite{yevick14}.
The effective-sphere model extends these capabilities
to include assessment of particle morphology without
sacrificing speed or ease of use.

\section{Lorenz-Mie characterization}
\label{sec:lorenzmiecharacterization}

\begin{figure}[!t]
  \centering
  \includegraphics[width=0.9\columnwidth]{hvm_schematic1}
  \caption[Schematic representation of Lorenz-Mie characterization]
  {(a) Lorenz-Mie characterization.  A colloidal sample
    flowing down a microfluidic channel is illuminated by a collimated
    laser beam.  Light scattered by a colloidal particle in the stream
    is collected by an objective lens and projected by a tube lens onto the sensor of
    a video camera, where it interferes with the unscattered portion
    of the beam to create a hologram.  (b) The experimentally recorded hologram for a
    typical colloidal polystyrene aggregate.  (c) Fit of the hologram
    in (b) to the Lorenz-Mie prediction from Eq.~\eqref{eq:lorenzmie}.}
  \label{fig:hvmschematic}
\end{figure}

Lorenz-Mie characterization, depicted schematically in
Fig.~\ref{fig:hvmschematic}(a),
is based on in-line holographic video microscopy \cite{sheng06},
in which the sample is illuminated with a collimated laser beam.
Light scattered by a particle interferes with the remainder of the
illumination in the focal plane of a microscope.
The intensity $I(\vec{r})$ of the magnified interference pattern is
recorded with a conventional video camera for analysis.
A typical example is shown in Fig.~\ref{fig:hvmschematic}(b).
Each holographic snapshot is corrected for the camera's
dark count, $I_d$, and is normalized by the
background intensity in the field of view, $I_0(\vec{r})$,
to obtain \cite{lee07,lee07a,krishnatreya14}
\begin{equation}
  \label{eq:normalization}
  b(\vec{r}) = \frac{I(\vec{r}) - I_d}{I_0(\vec{r}) - I_d}.
\end{equation}
The normalized hologram then is fit to the prediction
\cite{lee07a}
\begin{equation}
  \label{eq:lorenzmie}
  b(\vec{r}) 
  = 
  \abs{\hat{x} 
    + 
    e^{-i k z_p} \vec{f}_s(k(\vec{r} - \vec{r}_p)\vert a_p, n_p)}^2,
\end{equation}
where $k$ is the wavenumber of light in the medium,
$\vec{r}_p$ is the position of the particle's center relative
to the center of the microscope's focal plane, and
$\vec{f}_s(k \vec{r} \vert a_p, n_p)$ is the Lorenz-Mie function
that describes scattering of the incident wave by a sphere of radius $a_p$
and refractive index $n_p$ \cite{bohren83,mishchenko02}.
The form of Eq.~\eqref{eq:lorenzmie} is appropriate for a beam
propagating along $\hat{z}$ that is linearly polarized along $\hat{x}$.
Fitting Eq.~(\ref{eq:lorenzmie}), pixel by pixel,
to the normalized hologram of a sphere yields the sphere's three-dimensional position, 
its radius and its refractive index \cite{lee07a}.
Fig.~\ref{fig:hvmschematic}(c) presents the result of fitting to the
experimental hologram from Fig.~\ref{fig:hvmschematic}(b).

The custom-built holographic microscope used for this study
illuminates the sample with the collimated beam from
a solid state laser (Coherent Cube)
operating at a vacuum wavelength of $\lambda = \SI{447}{\nm}$.
The sample flows through the observation volume in 
microfluidic channel created by bonding
the edges of a number 1.5 glass microscope cover slip to the face of a
standard glass microscope slide.
This sample cell is mounted on a translation stage (Prior, Proscan II)
in the focal plane of an oil-immersion objective lens
(Nikon, Plan Apo, $100\times$, numerical aperture 1.45).
Light collected by the objective lens is relayed by an achromatic
tube lens to a video camera (NEC, TI-324AII), which records its
intensity 30 times per second with an effective magnification
of \SI{135}{\nm\per pixel}.
Each video frame is a \SI{640 x 480}{pixel} measurement of $I(\vec{r})$
with a resolution of \SI{8}{bits\per pixel}.
Features associated with dispersed particles are identified in
digitized holographic images \cite{krishnatreya14a} 
and are analyzed with Eqs.~\eqref{eq:normalization} and \eqref{eq:lorenzmie}
using methods that have been described in detail elsewhere
\cite{yevick14,krishnatreya14}.
The example in Fig.~\ref{fig:hvmschematic}(b) is a 
\SI{201 x 201}{pixel} region of interest cropped from the
normalized hologram, $b(\vec{r})$, obtained from $I(\vec{r})$
according to Eq.~\eqref{eq:normalization}.

All principal results were reproduced using a second instrument based 
on a $40\times$ air objective 
(Nikon Plan Fluor, numerical aperture 0.75)
operating at a vacuum wavelength of \SI{532}{\nm} 
(Thorlabs, CPS532 \SI{4.5}{\milli\watt})
with an effective magnification of \SI{120}{\nm\per pixel}
on an Allied Vision Mako U-130B camera.
This camera yields \SI{1280 x 1024}{pixel} images with
\SI{8}{bits \per pixel}.
Samples flow through this instrument in prefabricated microfluidic
channels with \SI{100}{\um} path length (Ibidi, $\mu$Slide VI,
uncoated).
Flow is driven by a syringe pump (New Era Systems, NE 100).
This reduces the possibility that instrumental artifacts
might have influenced the scaling relationships reported here.

A colloidal sample is characterized by placing a \SI{100}{\micro\liter}
aliquot in the reservoir at one end of the microfluidic channel and
drawing it through with a small pressure gradient.
The resulting Poiseuille flow has a peak speed along its axis 
of $v = \SI{150}{\um\per\second}$, which is small enough
to avoid artifacts due to motion blurring \cite{cheong09,dixon11}
and allows each particle to be recorded several times
during its transit.
Given a concentration on the order of
\SI{e7}{particles\per\milli\liter},
a few thousand particles will pass through the observation volume
in \SI{5}{\minute}.

\begin{figure}[!b]
  \centering
  \includegraphics[width=0.8\columnwidth]{track1_5}
  \caption[Tracking a colloidal polystyrene aggregate]
  {Tracking and characterizing a colloidal polystyrene
    aggregate.  (a) Eight holographic snapshots at \SI{1/15}{\second}
    intervals of an aggregate moving from the bottom of the microscope's field of
    view to the top at $v = \SI{120}{\um\per\second}$.  
    The scale bar represents \SI{20}{\um} in the imaging plane.
    (b) Fit values for the radius, $a_p$, and refractive index,
    $n_p$, obtained from the sequence of sixteen
    holograms recorded at \SI{1/30}{\second} intervals during the
    aggregate's \SI{0.5}{\second} transit.  Error bars
    represent uncertainties in the fit values.  Results from the
    images in (a) are plotted as circles and are interleaved with
    intervening results that are plotted as squares.
    Symbols are colored
    by time, as indicated by circles superimposed on the images.}
  \label{fig:track}
\end{figure}

A single snapshot of an individual colloidal particle can be analyzed
in several milliseconds using standard computer hardware
\cite{cheong09,yevick14}.
Characterization data therefore can be acquired in real time
as particles flow down the microfluidic channel \cite{cheong09}.
The images in Fig.~\ref{fig:track}(a) show eight stages of the transit
of a typical polystyrene aggregate at \SI{1/15}{\second} intervals.
The resulting time series of position and characterization data 
can be linked into a trajectory using maximum
likelihood methods \cite{crocker96,cheong09}.
The scatter plot in Fig.~\ref{fig:track}(b) shows the 
estimated values for the radius and refractive index obtained 
at each stage of the aggregate's
trajectory, recorded at \SI{1/30}{\second} intervals.
These results can be combined
into a trajectory-averaged estimate for the associated particle's characteristics.

When this measurement technique is applied to 
spherical particles, the standard deviation of the
trajectory-averaged characteristics is comparable to the
single-measurement precision \cite{cheong09,krishnatreya14}.
Results for the irregularly-shaped aggregate in Fig.~\ref{fig:track}
vary more substantially.
The standard deviation of the radius,
$\Delta a_p = \SI{0.04}{\um}$ is a factor of 10 larger
than the single-measurement precision.
The standard deviation of the refractive index, by contrast,
is comparable to the single particle precision,
$\Delta n_p = \num{0.002}$.
It is possible that the variation in apparent size occurs because
the aggregate is irregularly shaped, tumbles as it travels, and
so presents different-sized projections to the instrument.
We develop this interpretation in the next section.

\section{Effective sphere model}
\label{sec:effectivespheremodel}

Rather than attempting to generalize Eq.~(\ref{eq:lorenzmie})
to account for the detailed structure of
random colloidal aggregates, we instead
analyze their holograms with Eq.~(\ref{eq:lorenzmie}) itself
and interpret the results with effective medium theory
\cite{aspnes82,choy99}.
The radius, $a_p^\ast$, and refractive index,
$n_p^\ast$, obtained from such a fit then characterize
an effective sphere enclosing both the fractal aggregate
and also the intercalated fluid medium.
The goal of the present work is to establish a relationship
between the measured properties of the effective sphere and
the underlying properties of the actual aggregate.
This relationship then enables Lorenz-Mie microscopy to
probe the morphology of fractal
aggregates without incurring the computational burden of
detailed modeling.

For simplicity,
we assume that a fractal aggregate of fractal dimension $D$
is composed of identical spherical monomers, each of radius $a_0$
and refractive index $n_0$.
The aggregate is immersed in a medium of refractive index $n_m$
that fills the pores.
Provided that both the monomers and the pores are
substantially smaller than the wavelength of light,
this composite structure may be modeled as a continuous
medium whose refractive index, $n_p$, is given by the
Maxwell Garnett relation \cite{aspnes82},
\begin{subequations}
  \label{eq:gradientindex}
\begin{equation}
  L(n_p) = \phi_p \, L(n_0),
\end{equation}
where $\phi_p$ is the volume fraction of monomers in the sphere and
\begin{equation}
  L(n) = \frac{n^2 - n_m^2}{n^2 + 2 n_m^2}
\end{equation}
is the Lorentz-Lorenz factor.
\end{subequations}

The number of monomers within radius $r$ of a fractal aggregate's
center is
\begin{equation}
  \label{eq:number}
  N(r) = \left(\frac{r}{a_0}\right)^D, 
\end{equation}
and the associated volume fraction is
\begin{equation}
  \label{eq:internalvolumefraction}
  \phi(r) = \frac{a_0^3}{r^3} N(r) = \left(\frac{r}{a_0}\right)^{D-3}.
\end{equation}
For a particle of radius $a_p$, the overall volume fraction is
\begin{equation}
  \label{eq:volumefraction}
  \phi_p \equiv \phi(a_p) = \left(\frac{a_p}{a_0}\right)^{D-3}.
\end{equation}
From this and the Maxwell Garnett relation, we obtain an
expression for the cluster's effective refractive index,
\begin{equation}
  \label{eq:np}
  n_p
  =
  \sqrt{
    \frac{1 + 2 L(n_0) \phi_p}{1 - L(n_0) \phi_p}
  }.
\end{equation}
This result also may be expressed as a scaling 
relationship between the radius of a fractal aggregate and 
its effective refractive index,
\begin{equation}
  \label{eq:scalingprediction}
  \ln L(n_p) - \ln L(n_0)
  =
  (D - 3) 
  \ln\left(\frac{a_p}{a_0}\right).
\end{equation}
In a population of aggregates grown under comparable
conditions, Eq.~\eqref{eq:scalingprediction}
can be used to estimate the population-averaged
fractal dimension, $D$.

Equation~\eqref{eq:np} treats a cluster as a
homogeneous medium.  In fact, the density of monomers
decreases with scale, and therefore with radius within the
cluster.  As discussed in the Appendix, the associated 
radial gradient in the refractive index has little influence 
on practical measurements of clusters' effective characteristics.
We, therefore, ignore the clusters' 
spatial inhomogeneity in the discussion that follows.

\section{Comparison between effective medium theory formulas}

As discussed Chapter~\ref{ch:EMT}, there are three equations available for calculate the  effective refractive index for the fractal aggregate. We choose to use Maxwell Garnett formula in this work because each aggregate is a fractal structure consisting of nanometer scale polystyrene spheres embedded in the medium. For the experimental parameters used in this work, however, the numerical difference of the three formulas is close to the resolution of holographic microscopy. 

\begin{figure}[!t]
  \centering
  \includegraphics[width=0.6\textwidth]{emt3}
  \caption[Effective medium theory comparison]
    {The ratio between the effective refractive indexes of polystyrene aggregates 
    calculated from different formulas, as following: the ratio of (a) Lorentz-Lorenz 
    relation, (b) Maxwell Garnett relation using polystyrene particle as the host medium 
    and (c) Bruggeman relation to Maxwell Garnett relation using water as the host 
    medium, which is the calculation used in Section.~\ref{sec:effectivespheremodel}.}
  \label{fig:emt3}
\end{figure}

Fig.~\ref{fig:emt3} shows the ratio of three alternative effective medium modeling to the 
Maxwell Garnett relation we used in Section.~\ref{sec:effectivespheremodel}. The three 
cases are: (1) the Lorentz-Lorenz equation; (2) the Maxwell Garnett 
relation for an inverted "aggregate" of water in polystyrene; 
(3) the Bruggeman relation. 
In all calculations, $\phi_p$ represents the volume fraction of polystyrene particles, and $1-\phi_p$ is the volume fraction of water 
contained in the effective sphere. 
The value of refractive indexes for polystyrene and water are all same in each 
case, where $n_0 = 1.59$ and $n_m = 1.34$. The largest discrepancy is from Lorentz-Lorenz equation. The two methods differ by 0.2\% at volume fraction $\phi_p = 0.5$. The main peak in the size distribution of aggregate is at $a_p = \SI{600}{\nm}$. Using the fractal dimension $D = 1.75$, the volume fraction of polystyrene particles in the main population of aggregates is $\phi_p = (\SI{600}{\nm}/\SI{80}{\nm}) ^ {(1.75 - 3)} = 0.08$. This volume fraction corresponds to a 0.06\% difference in the effective refractive index, which is less than one part per thousand in the value of refractive index. The difference is beyond the resolution of the technique, which suggests that the choice of effective medium theory formula does not affect our analysis.

\section{Effective-sphere model for stratified spheres}
\label{sec:gradientindexspheres}

The effective-sphere model 
treats an aggregate as if the monomers were distributed uniformly
within it.
In fact, the marginal volume fraction decreases with distance $r$ 
from the center of the aggregate as
\begin{equation}
  \label{eq:surfacevolumefraction}
  \phi_s(r) = \frac{\frac{4}{3} \pi a_0^3}{4 \pi r^2} \frac{dN}{dr}
  =  \frac{D}{3} \left( \frac{r}{a_0} \right)^{D - 3} .
\end{equation}
This corresponds to a radial variation of the effective refractive index
described by
\begin{equation}
  \label{eq:grip}
  n(r) = n_m \, \sqrt{\frac{1 + 2 L(n_0) \phi_s(r)}{1 - L(n_0) \phi_s(r)}}.
\end{equation}
Unless the entire aggregate is smaller than the wavelength of light,
this radial structure might be expected to influence results 
obtained by applying effective medium theory to holograms of fractal aggregates.

To assess this influence,
we use the effective-sphere model to analyze synthetic holograms
of gradient-index particles with refractive-index profiles
described by Eq.~\eqref{eq:grip}.
These holograms are computed by replacing
$\vec{f}_s(k\vec{r}\vert a_p, n_p)$ in
Eq.~\eqref{eq:lorenzmie} 
with the corresponding generalized Lorenz-Mie 
result for a stratified sphere \cite{yang03,pena09,gouesbet11}
whose layers have refractive indexes given by Eq.~\eqref{eq:grip}.
The number of layers is chosen to converge the computed intensities
to within \SI{1}{\percent} at each pixel.
This typically occurs with layer thicknesses comparable
to $\pi / (5 k)$.
A similar approach has proved successful for Luneburg spheres
and other particles with continuous radial refractive index profiles
\cite{selmke15}.
The hologram is then fit to the Lorenz-Mie model
from Eq.~\eqref{eq:lorenzmie} for the effective sphere's radius,
$a_p^\ast$, and refractive index, $n_p^\ast$.
These parameters then can be compared with the true radius of
the stratified sphere, $a_p$, and the sphere-averaged refractive
index, $n_p$,
obtained from Eq.~\eqref{eq:scalingprediction}.

\begin{figure}[!t]
  \centering
  \includegraphics[width=1.0\textwidth]{stratified3}
  \caption[Effective-sphere model for gradient-index spheres]
  {Performance of the effective-sphere model for
    gradient-index spheres.
    (a) Effective radius, $a_p^\ast$, as a function of the stratified
    sphere's outer radius, $a_p$.  The dashed diagonal line
    indicates ideal agreement.
    (b) Effective refractive index, $n_p^\ast$, as a function of
    sphere radius.  The dashed curve indicates the sphere-averaged
    refractive index, $n_p$.
    (c) Scaling plot of the data, plotted according to
    Eq.~\eqref{eq:scalingprediction}.  The dashed diagonal
    line is obtained for the true values of radius and refractive
    index.}
  \label{fig:stratified}
\end{figure}

Fig.~\ref{fig:stratified} shows the performance 
of the effective-sphere model 
for particles with $D = \num{1.75}$, $a_0 = \SI{80}{\nm}$ and $n_0 = \num{1.585}$
in a medium with $n_m = \num{1.340}$.
These parameters are chosen to model fractal polystyrene aggregates
dispersed in water \cite{majolino89,zhou91,wu13}.
The effective sphere's radius, Fig.~\ref{fig:stratified}(a), 
and refractive index, Fig.~\ref{fig:stratified}(b),
both track the true values, albeit with systematic offsets.
They quite closely satisfy the anticipated scaling form predicted
by Eq.~\eqref{eq:scalingprediction} with a slope consistent
with the input fractal dimension, as can be seen in
Fig.~\ref{fig:stratified}(c).

Comparable results are obtained with different values of the
fractal dimension.
The gradient-index structure of fractal aggregates therefore does
not substantially diminish the ability of the effective-sphere model
to estimate such particles' fractal dimension.
Although the gradient-index model does not address the influence
of real aggregates' branched structure, it lends additional confidence
to the proposal that Lorenz-Mie characterization usefully assesses
the properties of such objects.

\section{Experimental studies of fractal aggregates of polystyrene nanospheres}
\label{sec:psaggregates}

The data in Fig.~\ref{fig:fractal} were obtained for \num{2727}
colloidal fractal aggregates grown under conditions conducive
to diffusion-limited cluster aggregation (DLCA) \cite{meakin88,majolino89,zhou91}. Each polystyrene nanoparticle is in Brownian motion. When two particles colloid and fall into the Van der Waals potential well, they bind rigidly into a particular configuration. The aggregates also diffuse randomly and bind to each other when they collide.
The primary particles are monodisperse polystyrene spheres with
a mean radius of $a_0 = \SI{80}{\nm}$ 
(Thermo Scientific, catalog number 5016A, 10\% w/w).

Aggregation was initiated by dispersing these particles in
\SI{0.5}{M} NaCl solution at a concentration of
\num{5e-5} by weight.
After one hour, the dispersion was diluted by a factor of \num{20}
with deionized water to stop further aggregation.
The resulting sample was then analyzed immediately, before
the aggregates might have time to restructure \cite{aubert86,zhou91}.

Each data point in Fig.~\ref{fig:fractal}(a) represents the
characteristics of a single colloidal particle obtained from a
trajectory such as the example in Fig.~\ref{fig:track}.
The plot symbols'
size is comparable to the numerical uncertainty in the fit parameters.
Each point is colored according to the relative probability density,
$P(a_p^\ast,n_p^\ast)$, of measurements according to the color bar
inset into Fig.~\ref{fig:fractal}(a).

The effective sphere model works well for aggregates with radii
smaller than \SI{2}{\um}.
Despite their irregular shapes, these objects are small enough
that their holograms display the radial symmetry typical of
spheres.
This can be seen in Fig.~\ref{fig:fractal}(b).
The reduced $\chi^2$ statistics for fits such as
Fig.~\ref{fig:fractal}(c) typically
fall within ten percent of unity, suggesting that the fit
parameters reliably reflect the aggregates properties.

\begin{figure}[t!]
  \centering
  \includegraphics[width=0.6\columnwidth]{fractal6a}
  \caption[Characterization of polystyrene fractal aggregates]
    {Characterization data for a population of polystyrene
    fractal aggregates. (a) Scatter plot of the effective radius,
    $a_p^\ast$, and refractive index, $n_p^\ast$, of \num{2727}
    aggregates, including the example from Fig.~\ref{fig:hvmschematic}(a).
    Each plot symbol reflects the properties of one aggregate, and
    is colored by the relative probability density of measurements, 
    $P(a_p^\ast,n_p^\ast)$.  The solid (blue) curve is the prediction of
    Eq.~\eqref{eq:scalingprediction} using the fractal dimension
    $D = \num{1.75}$ for diffusion-limited cluster aggregation,
    and no other adjustable parameters.
    (b) The same data replotted for comparison with the scaling
    prediction from Eq.~\eqref{eq:scalingprediction}.
    According to the effective-sphere model, the rescaled data are
    expected to fall along the solid (blue) line.
    Inset: Scanning electron microscope image of a typical aggregate.
    Scale bar indicates \SI{1}{\um}.
  }
  \label{fig:fractal}
\end{figure}

Larger aggregates are
reliably detected and counted by the feature identification algorithm \cite{krishnatreya14a},
but are poorly characterized by the effective-sphere model
\cite{wang16}.
Their holograms are more substantially asymmetric, and the
reduced $\chi^2$ statistic for these fits typically exceeds
\num{10}.
It is not surprising, therefore, that the estimated
characteristics for aggregates with $a_p > \SI{2.5}{\um}$
do not follow the trend expected for fractal aggregates, their refractive indexes
falling below the scaling prediction.

The solid curve in Fig.~\ref{fig:fractal}(a) shows the prediction from
Eq.~\eqref{eq:scalingprediction} for $n_p(a_p)$, with no adjustable
parameters.
In addition to the monomers' radius, the effective-sphere model
is parametrized by the monomers' refractive index and the
clusters' fractal dimension.
The former, $n_0 = \num{1.59(1)}$, was obtained from
holographic characterization studies
of emulsion-polymerized polystyrene spheres
\cite{lee07a,cheong09,xiao10a,cheong11}.
The latter, $D = \num{1.75(3)}$, was obtained from 
independent light-scattering
studies on aggregates grown under comparable conditions
\cite{aubert86,majolino89,zhou91,sorensen01,wu13}, and
is consistent with expectations for DLCA \cite{weitz85,sorensen01}.

Fig.~\ref{fig:fractal}(b) shows the same data replotted to emphasize
the scaling prediction from Eq.~\eqref{eq:scalingprediction}.
Agreement with the effective sphere model is quite good
for particles with apparent radii smaller than 
$a_p^\ast = \SI{2}{\um}$.
Because fractals' pore size increases with scale, larger aggregates
presumably do not satisfy the requirements of effective medium
theory and so are not so well described by the effective-sphere model.

The scanning electron microscope image inset into
Fig.~\ref{fig:fractal}(b) shows a typical vacuum-dried
aggregate.  This image resolves the individual spheres,
whose arrangement is consistent with the irregular branched
structure inferred from holographic characterization of
similar samples.  Although details of the structure undoubtedly
were altered during drying, the presence of voids at multiple
scales within the cluster is consistent with a fractal dimension
smaller than \num{2}.

The model's success for smaller clusters supports the contention
that Eq.~\eqref{eq:scalingprediction} can be useful
for measuring the population-averaged fractal
dimension of micrometer-scale fractal clusters.
We next apply this approach to characterize aggregates of
two model proteins whose cluster morphologies have
been independently established.
This application not only serves to verify the effective-sphere
model for fractal clusters, 
but also illustrates the utility of Lorenz-Mie
microscopy for measuring the size distribution and morphology
of protein aggregates, a subject of considerable interest in
biology \cite{morris09}
and of substantial practical importance in pharmaceutical manufacturing \cite{wang99a}.

\section{Experimental studies of protein aggregates}
\label{sec:protein}

\begin{figure}[!t]
  \centering
  \includegraphics[width=0.6\columnwidth]{protein4}
  \caption[Scaling prediction of protein aggregates]
    {Scaling prediction of Eq.~\eqref{eq:scalingprediction} for the Lorenz-Mie 
    characterization of insulin aggregates showed in Fig.~\ref{fig:hologram} and 
    BSA aggregates results combined from Fig.~\ref{fig:bsa_runs}(a) and Fig.~
    \ref{fig:bsa_runs}(b).
    Radii are measured relative to the arbitrary scale, $a_0 = \SI{1}{\um}$.
    Each point represents the properties of a single colloidal
    particle and is colored according to the local density of
    measurements.
    Superimposed (red) lines corresponds to the best-fit
    fractal dimension for each sample, with dashed lines indicating 
    a range of \num{+-0.1}.
    (a) Bovine pancreas insulin.  
    The data are consistent with $D = \num{1.5(1)}$.
    (b) Bovine serum albumin.  The distribution is consistent with 
    $D = \num{1.1(1)}$.}
  \label{fig:protein}
\end{figure}

Like the colloidal nanoparticles considered in the previous section,
proteins in solution also have a tendency to aggregate \cite{morris09}.
Some of the resulting macromolecular structures perform important biological
functions.  Others cause diseases.
Protein aggregation is a principal failure mechanism for biopharmaceutical
formulations not only because clustered proteins are less effective as
therapeutic agents, but also because they can elicit dangerous immune
responses \cite{wang99a}.
We previously have demonstrated that inline holographic video 
microscopy can detect protein aggregates in solution, and can
distinguish them from such common contaminants as silicone oil
droplets \cite{wang16}.
Here, we apply the effective-sphere model to study the aggregates'
morphology by estimating their fractal dimension.

\subsection{Bovine insulin}

The data in Fig.~\ref{fig:protein}(a) are the scaling prediction of 
Eq.~\eqref{eq:scalingprediction} for the Lorenz-Mie characterization of insulin aggregates showed in Fig.~\ref{fig:hologram}.
Because the effective monomer radius is not known \emph{a priori},
radii in Fig.~\ref{fig:protein} are scaled by an arbitrary factor,
$a_0 = \SI{1}{\um}$.  This choice does not affect the
estimate for $D$.

The main distribution of single-particle characteristics follows the
scaling prediction quite well, and has a slope consistent with
a fractal dimension of $D = \num{1.5}$.  This is denoted in
Fig.~\ref{fig:protein}(a) by a solid (red) line superimposed on the
data.

Dashed lines in this plot show equivalent results for fractal
dimensions $D = \num{1.4}$ and $D = \num{1.6}$.
The proposal that bovine insulin forms branched fractal aggregates
is consistent with independent measurements of such aggregates'
morphology using atomic-force microscopy \cite{siposova12}.

In addition to the main distribution of points,
Fig.~\ref{fig:protein}(a) features an outlying cluster of large-size
aggregates comparable to those in Fig.~\ref{fig:fractal}.
It also includes a cluster of small particles
with low refractive indexes.
These latter features appear to correspond to 
globular aggregates that are distinct from the
fractal clusters of interest here.

\subsection{Bovine serum albumin}

The data in Fig.~\ref{fig:protein}(b) are the scaling prediction of Eq.~\eqref{eq:scalingprediction} for the Lorenz-Mie characterization of BSA aggregates results combined from Fig.~\ref{fig:bsa_runs}(a) and Fig.~\ref{fig:bsa_runs}(b).
The results for BSA aggregates with radii smaller than \SI{2}{\um}
agree well with the scaling prediction from
Eq.~\eqref{eq:scalingprediction},
this time with an apparent fractal dimension of $D = \num{1.1}$,
as indicated by the solid (red) line superimposed on the
data in Fig.~\ref{fig:protein}(b).
The nearly linear structure suggested by this low fractal dimension
is consistent with atomic force microscopy images of 
BSA aggregates \cite{omichi14}.

\section{Conclusions}

The results 
presented in this chapter demonstrate that Lorenz-Mie microscopy can provide useful insights
into the properties of micrometer-scale fractal aggregates.
Holograms of micrometer-scale colloidal fractal
aggregates can be interpreted with
the effective-sphere model presented in
Sec.~\ref{sec:effectivespheremodel}
to estimate an aggregate's size
and effective refractive index.
These particle-resolved data, in turn, can be pooled
to estimate the population-averaged fractal dimension.
The effective-sphere model therefore extends
the particle-characterization capabilities of
Lorenz-Mie microscopy to irregularly branched
objects.

The success of the effective-sphere model in 
characterizing ramified protein aggregates lends
additional support to the earlier proposal \cite{wang16}
that Lorenz-Mie characterization meaningfully assesses
such aggregates' sizes.
It therefore establishes Lorenz-Mie microscopy as a
method for sizing protein aggregates, characterizing
their morphology, and differentiating them from 
other types of colloidal particles.

This work also provides a baseline against which
more detailed approaches to holographic characterization
of fractal structures may be compared.
Future extensions based on machine-learning techniques \cite{yevick14}
or direct modeling of the spatial distribution of 
dielectric material \cite{fung12} thus can be tested directly using
the methods described here.