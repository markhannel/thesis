\chapter{Conclusions}
\label{ch:conclusion}

This thesis validates, extends, and employs holographic
particle characterization. 

\autoref{ch:debye} validates the use of the scalar theory
approximation for holographic particle characterization
with the Lorenz-Mie theory. By fitting synthetic holograms
based on a vector-based theory to a scalar theory,
our simulations determine that for scatterers between the
sizes of \% insert range, the scalar theory is offers
the same prediction as a vector-based theory, albeit
with a less computational overhead. 

Fitting holographic features to the Lorenz-Mie theory
assumes a homogeneous spherical scatterer and yet colloidal particles
come in many shapes such as aggregates, clusters and rods. In \autoref{ch:dimpled}
we present a series of experimental and simulated results


\autoref{ch:cascade}

\autoref{ch:svr}

\autoref{ch:synthesis}
