\chapter{Conclusions}
\label{ch:conclusion}

This thesis validates, extends, and employs holographic
particle characterization, a relatively new technique
for extracting the position, size, and composition
of micometer-scale scatterers \emph{in situ}. We
advance the technique by providing a
firm foundation for the scalar-wave approximation
in image formation and introduces the effective
sphere model which expands its domain of applicability.
In addition we implement machine learning algorithms for
detecting, localizing, and characterizing holograms; these
advances greatly reduce the computation time and enable
the real-time analysis of dense, heterogeneous samples.
We then use the technique to investigate the role of
environment conditions and chemical choices in
a specific emulsion polymerization process.

In \autoref{ch:debye} we justified the use of the scalar theory
approximation outlined in \autoref{ch:hvm}.
Where possible, we accounted for the vectorial nature
of light propagation including reflection, refraction,
angular demagnification, and polarization rotation
through the elements of the optical train. By fitting synthetic holograms
based on a vector-based theory to a scalar theory,
our simulations determined that for scatterers between \SI{0.5}{\um}
and \SI{10}{\um} with refractive index between \SI{1.4}{} and \SI{1.7}{},
the scalar theory provides the same prediction as a vector-based theory,
albeit with far less computational overhead. 

Fitting holographic features to the Lorenz-Mie theory
assumes a homogeneous, spherical scatterer and yet colloidal particles
come in many shapes such as aggregates, clusters, and rods. In \autoref{ch:dimpled}
we presented a series of experimental and simulated results that demonstrate
that the Lorenz-Mie theory provides a first-order approximation for light scattering
off of aspherical particles. For micrometer sized particles with small deviations
from a spherical geometry, fitting to the Lorenz-Mie theory yields
an effective size and refractive index measurement.

In \autoref{ch:cascade} we considered the first step in analyzing recorded
images: detecting and localizing holographic features. We trained two machine
learning algorithms, a cascade classifier and a convolutional neural network (CNN),
to accomplish these tasks. We compared these two techniques against previously
employed heuristics for precision, recall, and speed. Our tests demonstrate that
the CNN and cascade respectively boast speeds-ups of \num{10} and \num{100} times
faster processing than our heuristics. The CNN can robustly detect and accurately
localize highly overlapping, differing contrast features with exceedingly few false positives.
The Haar-based cascade classifier provides single-pixel rather than
sub-pixel precision yet its speed and modest computational requirements provide
real-time localization and detection abilities. We demonstrated that the cascade
classifier can operate as a high-speed targeting system for a holographic
optical trapping system where single-pixel precision suffices to capture
rapidly diffusing targets.

Fitting holographic features to the Lorenz-Mie theory is computationally intensive
and sensitive to initial estimates of the particle's parameters. For heterogeneous
samples with disparate properties, a single initial estimate may not suffice
to accurately measure each scatterer's properties. In \autoref{ch:svr} we
addressed these pain points by analyzing holographic features with a trained set
of three support vector machines which estimate the diameter, refractive index
and distance from the focal plane, respectively. Each support vector
machine is provided a hologram's radial profile, the intensity averaged over
polar angles, and yields an estimate of the particle's properties. Although
this analysis yields one-tenth the precision and accuracy of non-linear fitting,
it proceeds one-thousand times faster. We experimentally determined that
support vector machines can easily analyze a dense suspension of four
disparate particle types. For assays requiring the full precision of non-linear
fitting, support vector machines can therefore bootstrap subsequent fitting with sufficiently
close initial estimates and enable detailed analysis of heterogeneous samples.

We conclude our work in \autoref{ch:synthesis} by demonstrating the use of holographic
particle characterization in investigating protocol choices in the synthesis of
3-(trimethoxysilyl)propyl methacrylate (TPM) colloidal spheres.
The synthesis of TPM particles is a two-step process: emulsion formation and subsequent
polymerization. Stirring during emulsion formation maintains well-mixed conditions for
nucleating TPM droplets and keeps the sample from phase separating. We determined
that increasing the stir rate initially increases average particle diameter
before yielding polydisperse samples. These holographic characterization results
of diameters for polymerized spheres were corroborated by SEM analysis. Additionally,
our refractive index measurements were corroborated by Abbe refractometry of many
dilute suspensions of TPM spheres. Our holographic results were also able to measure
the diameters and refractive indexes of TPM liquid droplets and demonstrated the large
increase in refractive index between monomeric TPM oil and oligomerized liquid TPM.
By periodically sampling a sample undergoing heat-activated polymerization, we
determined that the polymerization process ends after approximately \SI{20}{\minute}.
Finally, we synthesized four emulsions of liquid droplets with
binary choices in TPM oil and ammonium chloride concentration. Each of the four
emulsions were then polymerized with four initiators, two water-soluble and two
water-insoluble, to produce a total of sixteen polymerized spheres.
These holographic assays reveal no dependence between particle properties and initiator choice;
they do, however, uncover a complex relationship between the resulting sphere diameters
and stoichiometry. For low concentrations of ammonium chloride,
an increase in TPM added will create larger droplets. For high concentrations of ammonium
chloride, an increase in TPM will generate more droplets.




