\chapter{Conclusions}
\label{ch:conclusion}


This thesis validates, extends, and employs holographic
particle characterization, a relatively new technique
for extracting the position, size, and composition
of micometer-scale scatterers \emph{in situ}. This
thesis advances the technique by providing a
firm foundation for the scalar-wave approximation
in image formation and introduces the effective
sphere model which expands its domain of applicability.
In addition we implement machine learning algorithms for
detecting, localizing, and characterizing holograms; these
advances greatly reduce the computation time and enable of
the real-time analysis of dense, heterogeneous samples.
We then use the technique to investigate the role of
environment conditions and chemical choices in
a specific emulsion polymerization process.

Our work in validating the use of the scalar theory
approximation is provided \autoref{ch:debye}.
Where possible, we account for the vectorial nature
of light propagation including reflection, refraction,
angular demagnification, and polarization rotation
through the elements of the optical train. By fitting synthetic holograms
based on a vector-based theory to a scalar theory,
our simulations determine that for scatterers between \SI{0.5}{\um}
and \SI{10}{\um} with refractive index between \SI{1.4}{} and \SI{1.7}{},
the scalar theory is provides the same prediction as a vector-based theory,
albeit with far less computational overhead. 

Fitting holographic features to the Lorenz-Mie theory
assumes a homogeneous spherical scatterer and yet colloidal particles
come in many shapes such as aggregates, clusters and rods. In \autoref{ch:dimpled}
we present a series of experimental and simulated results that demonstrate
that assuming the particle is spherical yields accurate, sensible results.

In \autoref{ch:cascade} we consider the first step in analyzing recorded
images: detecting and localizing holographic features. We trained two machine
learning algorithms, a cascade classifier and a convolutional neural network (CNN),
to accomplish these tasks. We compared these two techniques against previously
employed heuristics for precision, recall, and speed. In all our testing
the CNN and cascade boast speeds-ups for \num{10} and \num{100} times
faster processing than our heuristics. 

\autoref{ch:svr}

\autoref{ch:synthesis}

