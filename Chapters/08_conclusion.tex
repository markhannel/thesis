\chapter{Conclusions}
\label{ch:conclusion}



This thesis validates, extends, and employs holographic
particle characterization, a relatively new technique
for extracting the position, size, and composition
of micometer-scale scatterers \emph{in situ}. This
thesis advances the technique by providing a
firm foundation for the scalar-wave approximation
in image formation and introduces the effective
sphere model which expands its domain of applicability.
In addition we implement machine learning algorithms for
detecting, localizing, and characterizing holograms; these
advances greatly reduce the computation time and enable of
the real-time analysis of dense, heterogeneous samples.
We then use the technique to investigate the role of
environment conditions and chemical choices in
a specific emulsion polymerization process.

In \autoref{ch:debye} we justified the use of the scalar theory
approximation is outlined in \autoref{ch:hvm}.
Where possible, we accounted for the vectorial nature
of light propagation including reflection, refraction,
angular demagnification, and polarization rotation
through the elements of the optical train. By fitting synthetic holograms
based on a vector-based theory to a scalar theory,
our simulations determined that for scatterers between \SI{0.5}{\um}
and \SI{10}{\um} with refractive index between \SI{1.4}{} and \SI{1.7}{},
the scalar theory is provides the same prediction as a vector-based theory,
albeit with far less computational overhead. 

Fitting holographic features to the Lorenz-Mie theory
assumes a homogeneous spherical scatterer and yet colloidal particles
come in many shapes such as aggregates, clusters and rods. In \autoref{ch:dimpled}
we presented a series of experimental and simulated results that demonstrate
that the Lorenz-Mie theory provides a first-order approximation for light scattering
off of aspherical particles. For micrometer sized particles with small deviations
from a spherical geometry, fitting to the Lorenz-Mie theory measures
an effective size and refractive index.

In \autoref{ch:cascade} we considered the first step in analyzing recorded
images: detecting and localizing holographic features. We trained two machine
learning algorithms, a cascade classifier and a convolutional neural network (CNN),
to accomplish these tasks. We compared these two techniques against previously
employed heuristics for precision, recall, and speed. Our tests demonstrate that
the CNN and cascade respectively boast speeds-ups of \num{10} and \num{100} times
faster processing than our heuristics. The CNN can robustly detect and accurately
localize highly overlapping, high-contrast features with exceedingly few false positives.
The Haar-based cascade classifier provides single-pixel rather than
sub-pixel precision yet its speed and modest computational requirements provide
real-time localization and detection abilities. We demonstrated that the cascade
classifiers can operate as a high-speed targeting system for a holographic
optical trapping system where single-pixel precision suffices to capture
rapidly diffusing targets.

\autoref{ch:svr}

\autoref{ch:synthesis}

