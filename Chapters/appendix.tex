
\subsubsection{Digitizing Images}
\label{ch:hvm:sec:hvm:ssec:digitalrec:sssec:digitizing}

Modern digital cameras employ an array of photon detectors to measure the
average intensity over the sensor surface. Each photon detector, referred to as a pixel,
utilizes the photoelectric effect to convert photons into excited electrons.
The excited electrons at each pixel are counted, sometimes to single precision, and
then digitized into $8$-, $12$-, or even $16$-bit integers. Because the number of
excited electrons is proportional to the number of photons, and the the number of
photons is in turn proportional to the intensity of the of the image at the
sensor surface, the array of electron counts serves as a proxy for the intensity
of the image.

With millions of pixels and the capacity to count tens of thousands of electrons at each pixel,
digital cameras are an engineering marvel. For our purposes, we will review a few of the physical
details of a single pixel to explain important imaging effects such as dark counts, saturation,
and the digitization procedure.

Pixels are designed to accurately count the number of electrons that are
excited by incident photons within a given exposure period. A number of physical constraints
limit the accuracy of each pixels.

During an exposure period, $N_p$ photons of a particular wavelength arrive at a pixel
surface. Some fraction of the incident photons are converted to excited electrons
with a wavelength dependent probability known as the quantum efficiency. To be properly
counted, these excited electrons must survive until the counting procedure has
accounted for their presence. To this end, each pixel is doped to increase the lifetime
of excited electrons. In addition, the excited electrons must remain in the bulk so that
they are not grounded; for this purpose, a biased field is applied. % FIXME: What about mirror charges?

During the exposure period, a number of electrons can be thermally excited (as opposed to
photonically excited) and will be included in the count. These erroneously
counted electrons are referred to as a dark count. Measuring the average dark count of a
camera is as easy as recording the average image while an opaque object is blocking the
camera's sensor. Note that the dark count will increase with the length of the exposure
period.

% The number of electrons that can be negative.. scientific cameras have a non-zero
% floor to maintain gaussian-errors.
% Saturation occurs because the relation number of excited electrons per
% number of incident photons becomes non-linear. The largest number of reported
% electrons is the highest level

% A number of approximations come to mind with the LM theory.
% Approximation of radial component
% Approximation of functional form (Hankel function)
% Approximation of polarization

