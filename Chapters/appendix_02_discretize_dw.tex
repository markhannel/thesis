\SkipTocEntry\chapter{Discretizing the scattered field for the Debye-Wolf integral} 
\label{app:discretize_dw}
\appcaption{Appendix B}

 \Autoref{ch:debye} presents a model for propagating the incident and
 scattered electric fields through the optical train to a digital camera.
 Eqs.~\eqref{eq:strength_factor} through \eqref{eq:magnification} account for the
 refraction, reflection, and angular demagnification of the electric
 field strength factor up to the tube lens.
 The Debye-Wolf integral, Eq.~\eqref{eq:debyewolf}, transforms the
 electric field strength factor at the tube lens into the electric field present in the
 imaging plane. We have implemented this transformation as a discrete Fourier
 transform as shown in Eq.~\eqref{eq:complete_dw}. This appendix describes an appropriate
 discretization of the Fourier transform such that the Debye-Wolf integral converges
 and the spacing in the image plane approximates the size of a pixel.
 A more thorough account of this discretization is provided in Ref.~\cite{capoglu12},
 section 3.4.2.

\section{Appropriately sampling the electric field strength factor}
  Figure~\ref{fig:debye_schematic} presents the discretization schemes for the
  three grids in our computation: plane \#2 and
  plane \#3, where we evaluate the scattered field's electric field strength factor, and
  plane \#4, where we evaluate the fields comprising the image.
  While our formalism allows for unequal spacing in the $x$- and $y$-direction,
  we assume that the spacings in each direction are identical and
  therefore drop subscripts denoting direction.
  The spacings in plane \#2,
  $\Delta s$, and plane \#3, $\Delta s^{\prime}$ are linearly
  related by the Abbe sine condition, Eq.~\eqref{eq:magnification};
  specifying one spacing dictates the other. The direction cosine, $s$, is bounded by the numerical
  aperture of the objective lens such that $\abs{s}\le \text{NA} /n$
  where $NA$ is the numerical aperture of the objective and $n$ is the refractive index
  the immersion oil.
  Sampling plane \#2 $P$ times therefore results in a spacing $\Delta s = 2\, \text{NA}/(n\,P)$.

  The scattered field has an extent of size $W$ in the focal plane of the objective
  wherein the amplitude is non-negligible. We impose the condition
  $\Delta s < \lambda/W$ to prevent aliasing and derive a condtion
  on the sampling number $P$ \cite{capoglu12},
  \begin{equation}
    P > \frac{2 \text{NA}\, W}{n\lambda}.
  \end{equation}
  
  The field $\vec{E}_{img}$ is evaluated on a grid with spacing $\Delta$
  in the imaging plane of the camera, plane \#4. Ideally the spacing in the image
  should well-approximate the real spacing between adjacent 
  pixels of the camera after magnification, $M$:
  \begin{equation}
  \Delta_x \approx M\, \text{mpp}.
  \end{equation}        
  While sampling the plane \#3 $P$ times, the above equation is overly constrained and
  can not in general be satisfied. We can, however, pad the electric field strength factor
  at plane \#3 with zeros: the numerical aperture of the system sets these values to
  zero. By padding the electric field strength factor with $p$ zeros,
  the Fourier transform in Eq.~\eqref{eq:complete_dw} will be effectively
  interpolating high frequency responses but will not actually increase or decrease
  the spatial resolution of the result. With zero padding, the spacing in the imaging
  plane is
  \begin{equation}
  \Delta = \frac{\lambda P}{\left ( P + p \right ) 2\, \text{NA}}.
  \end{equation} 
  The approximation $\Delta \approx \text{mpp} \, M$  necessitates that the
  padding $p$ must approximately satisfy
  \begin{equation}
    p \approx P \left ( \frac{\lambda}{2 \, \text{NA} \, \text{mpp}} - 1 \right ).
    \label{eq:padding_approx}
  \end{equation}
  Equation~\eqref{eq:padding_approx} can not in general be solved exactly as $p$
  must be an integer. For the cases we considered, a $p$ value of approximately
  one-tenth of $P$ led to a \SI{0.5}{\percent} mismatch between the pixel size
  and the spacing $\Delta$. If necessary, the image can be slightly interpolated
  or extrapolated to a grid corresponding to the real pixel locations. This
  approach was adopted the calculations in \autoref{ch:debye}.
  
