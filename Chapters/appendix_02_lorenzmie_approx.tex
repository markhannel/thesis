\SkipTocEntry\chapter{Approximations to the Lorenz-Mie Scattered fields} 
\label{app:lorenzmie_approx}
\appcaption{Appendix B}

The Lorenz-Mie theory underlies several techniques over including static light scattering
\cite{zimm48}, imaging through interstellar grains \cite{purcell1969} and sizing large
air bubbles \cite{hansen85}. In this appendix we examine the validity of
common approximations for the Lorenz-Mie scattering theory in the context of
holographic particle characterization.


\section{Sizing the scattered field's radial component}



The light scattered by a spherical particle, $\vec{E}_s$, propagates radially and has a
non-zero, albeit decaying, radial component, $\vec{E}_s\cdot\uvec{r}$.
For many applications, the scattered wave has propagated sufficiently far that the
radial component can be safely neglected. For holographic particle characterization,
the fields may be evaluated as close as the focal plane for the scalar theory
presented in \autoref{ch:hvm} and as far as the entrance pupil for the vector-based
theory presented in \autoref{ch:debye}.
We probe the radial component's contribution to the scattered field by measuring
the its fractional contribution, $\vec{E}_s\cdot\uvec{r} \, / \abs{\vec{E}_s}$,
over a range or particle sizes and refractive indexes. The results of this analysis are
presented in Fig.~\ref{fig:radial}.


