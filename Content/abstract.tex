\chapter*{Abstract}
\addcontentsline{toc}{chapter}{Abstract}
\label{ch:abstract}


Colloidal systems exhibit a range of interesting behaviors that
underlie many commercial and industrial applications, and continue
to gainfully employ academics. Often these systems are comprised of
micron-sized constituents whose properties, namely their
size, refractive index, and three-dimensional position, can be measured
by holographic video microscopy (HVM). This thesis validates the effective
sphere model for HVM, thereby demonstrating the efficacy of the
Lorenz-Mie model outside the domain of perfect spherical scatters;
furthermore we develop machine-learning implements for robust feature
detection and characterization, and consequently enable the analysis of
dense, heterogeneous samples; finally we apply this technique to
isolate the effect certain reaction environments have on the
synthesis of TPM spheres.

Despite the ubiquity of the spherical approximation, many colloidal
systems are compromised of aspherical particles such as rods,
aggregates, and dimpled spheres.
Given their departure from
ideal sphericity, to what extent can we model their light scattering
properties with the Lorenz-Mie model?
We experimentally demonstrate the effectiveness of the Lorenz-Mie model
in characterizing dimpled spheres:
should the dimple comprise less than \num{5}\% of the total sphere volume,
the resulting size, and position measurements incur
less than \num{1}\% error, while the refractive index suffers a modest \num{0.1}\%
error. Utilizing the discrete dipole
approximation, we simulate the scattering patterns of a myriad of
dimpled sphere geometries and find our experimental results are
typical for modest dimple sizes.
%As the effective sphere model
%represents a zeroth-order approximation for light scattering off
%non-spherical geometries, we expect that the Lorenz-Mie model can be
%applied generally to non-spherical scatterers with a not-too pronounced
%departure from ideal sphericity.

Heterogeneous samples pose a host of problems for automated
image analysis. Heuristics for detecting and localizing holographic
features typically employ a threshold which may overlook weaker
scatters or may identify several false positive in the vicinity of
overlapping features; dense, heterogeneous samples
particularly fall prey to these issues. To mitigate these constraints, we
develop two machine-learning implements, namely convolutional
neural networks and cascade classifiers, to reduce the number of
false negatives and false positives. After successfully isolating
features, it is necessary to provide a sufficiently close initial
estimate of the scatterer's properties. Given a heterogeneous sample
with disparate scatterer properties, a single {\it a priori}
estimate may not suffice. We demonstrate that support vector machines
provide estimates of a scatterer's size, refractive index, and
position that are not only sufficient for boot-strapping subsequent
Lorenz-Mie analysis, but are accurate enough for low-precision
applications.

Methods for synthesizing colloidal spheres are often sensitive
to a multitude of protocol choices and environmental conditions. We
utilize HVM to investigate the synthesis of 3-methacryloxypropyl
trimethoxysilane (TPM) spheres. Specifically, we synthesize numerous
batches of TPM spheres under varied conditions and measure the
resulting size and refractive index distributions. From this analysis,
we quantify and consequently corroborate the assertions of previous
work on TPM synthesis.


%HVM and pursuit of quantitative microscopy. Inherent assumptions
%and hang ups.
%This thesis validates and extends the domain of applicability for holographic characterization,
%and presents novel methods for analysis that enable new applications.

%Assumptions: Sphericity and vectorial nature of light propagation. Sphericity
% enables the effective sphere model.

%Analysis: Feature detection and localization. Enables analysis of denser flows, at a faster
%rate and with fewer false positives. Real time detection for automated trapping.
%ML for characterization. Automates and provides robust initial parameter estimation.
%Enables analysis of heterogeneous samples.

%Characterizing colloidal synthesis:  Utilizing HVM we will 
