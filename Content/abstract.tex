\chapter*{Abstract}
\addcontentsline{toc}{chapter}{Abstract}
\label{ch:abstract}


\fancypagestyle{abstract_style}
{
\fancyhf{}
\lhead{{\bf Title: }\thesistitle\\
  {\bf Author: } \thesisauthor\\
  {\bf Advisor: } \thesisadvisor}
}

%\thispagestyle{abstract_style}

Colloidal systems exhibit a range of interesting behaviors that
underlie many fundamental processes and industrial applications.
Often these systems are comprised of
micrometer-sized constituents whose properties, namely their
size, refractive index, and three-dimensional position, can be measured
by holographic particle characterization (HPC).
The availability of this wealth of particle-resolved information
has inspired research activities around the world, and has rapidly
been adopted by industry.
Previous implementations of HPC have relied on \emph{ad hoc}
models and simplifying assumptions to extract information from
recorded holograms.
This thesis provides a firm foundation for the technique both by
advancing the theory of image formation in holographic microscopy
and also by reporting a suite of validation measurements
that establish the limits of precision and accuracy that can be
obtained from measurements based on holograms.

A particular focus of this work is to introduce and validate
the effective sphere model that extends the benefits of holographic
characterization to aspherical particles without incurring
an exorbitant cost in computational complexity.
In developing the effective sphere model, we furthermore 
introduce machine-learning implementations for fast and robust feature
detection and characterization.  These techniques enable us to
analyze dense and heterogeneous samples that ordinarily are
not amenable to optical characterization.
Finally, we apply our techniques to guide the design of colloidal
synthesis protocols, with a particular application to 
monodisperse spheres of 3-(trimethoxysilyl) propyl methacrylate
(TPM).

The effective sphere model proves to be an effective approach
to assess the properties of colloidal
systems compromised of aspherical particles such as rods,
aggregates, and dimpled spheres.
This line of inquiry addresses the question:
``To what extent does the Lorenz-Mie theory for light scattering
by spheres apply to aspherical and inhomogeneous colloidal particles?''
As a model system with a well-defined departure from sphericity,
we experimentally apply the effective sphere model to
characterizing dimpled spheres.
These studies demonstrate that standard holographic characterization
techniques yield a particle's radius to within \SI{1}{\percent} error
and its refractive index to within \SI{0.1}{\percent}
when the dimple comprises less than \SI{5}{\percent} of sphere's total volume.
Utilizing the discrete dipole
approximation, we simulate the scattering patterns of a myriad of
dimpled sphere geometries and find our experimental results are
typical for modest dimple sizes.

Heterogeneous samples pose a host of problems for automated
image analysis. Heuristics for detecting and localizing holographic
features typically employ a threshold which may overlook weaker
scatters or may identify several false positive in the vicinity of
overlapping features; dense, heterogeneous samples
particularly fall prey to these issues. To mitigate these constraints, we
develop two machine-learning implementations, namely convolutional
neural networks and cascade classifiers, to reduce the number of
false negative and false positive defections. 
Further characterization of a detected particle's properties require
sufficiently close initial parameter estimates so that the applied
fitting algorithm can converge to a global minimum.
 Given a heterogeneous sample
with disparate scatterer properties, a single \emph{a priori}
estimate may not suffice. We demonstrate that a class of machine-learning
algorithms known as support vector machines can be trained to
provide estimates of a scatterer's size, refractive index, and
position that are not only sufficient for bootstrapping subsequent
Lorenz-Mie analysis, but are accurate enough in themselves
for many end-use applications.

Methods for synthesizing colloidal spheres are often sensitive
to a multitude of protocol choices and environmental conditions. We
utilize HPC to investigate the synthesis of TPM spheres.
Specifically, we synthesize TPM spheres under a wide array of
conditions and apply holographic particle characterization to measure the
resulting distributions of particle size and refractive index.
The unique wealth of information provided by holographic characterization
enables us to assess directly how processing protocols influence
particles' final properties.


%HVM and pursuit of quantitative microscopy. Inherent assumptions
%and hang ups.
%This thesis validates and extends the domain of applicability for holographic characterization,
%and presents novel methods for analysis that enable new applications.

%Assumptions: Sphericity and vectorial nature of light propagation. Sphericity
% enables the effective sphere model.

%Analysis: Feature detection and localization. Enables analysis of denser flows, at a faster
%rate and with fewer false positives. Real time detection for automated trapping.
%ML for characterization. Automates and provides robust initial parameter estimation.
%Enables analysis of heterogeneous samples.

%Characterizing colloidal synthesis:  Utilizing HVM we will 
