Chapter 2 big goals:

Introduce HVM from a theory perspective and an experimental perspective.
Give an intuition for why HVM works and best practices.

Outline the setup.

Theory Guidelines
 General scattering | Einc + Escat |^2
 - the mixed term has the relative phase information between
   the scattered and incident fields. This information is manifested
   as an interference pattern.

 Lorenz-Mie solution has a number of properties
 - For the particle's of interest, namely low refractive index, non-absorbative
 spheres, the scattered field primarily scatterers light forward. Under this
 assumption, the two E_s terms must average to zero owing to energy conservation.
 - The scattering coefficient have
 Mo1n and Ne1n are naturally functions of $n_m$ but not $a_p$ and $n_p$.. the basis
 for the vector spherical harmonics need not know about the scatterer. The scattering
 coefficients are in fact a function of $n_m$, $n_p$ and $a_p$. The scattering coefficients
 have

 - scatterer properties r_p, a_p, n_p
 - medium: n_m
 - Incident beam  $\lambda$
 - This solution works as is for absorptive media and particles and can easily
 extended to layered particles.


Sample prep.

Sample density needs to be in a certain range.
- Flow cell ~few particles per FOV
- Sealed Sample ~1 particle per many FOV's

Illumination.

Not too high.. you'll saturate the image.
Not too low... you want a high signal to noise ratio.

Automated fitting of holographic features.

Detect and Localize holographic features.
- Include caveats: overlapping holograms, aggregates, and other coherent scattering sources.
- Sub-pixel precision paradox
- 


TO DO:

   Make figure of the setup (hvm_setup)
   Introduce the LM theory before eq 2.5 more eloquently. Explain it is solving
   Maxwell's equation in the a charge/current-less medium with two boundary
   conditions
   Explain that the N,M do not know about the particle but do know
   about the position. An Bn know about the
   particle but not the position. Note that $a_p$ and $n_p$ are coupled,
   but not completely.
   Relegate too detailed discussion of digital recording to an appendix
   Write a section about image analysis
   Write the section about applications
   Figure to explain image normalization/holographic features.
