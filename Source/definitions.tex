%Packages to use
%%%%%%%%%%%%%%%%%%%%%%%%%%%%%%%%%%%%%%%%%%%%%%%%%%%%%%%%

\usepackage[nottoc]{tocbibind}
\usepackage[toc,page]{appendix}
% citations

\usepackage{cite}

\usepackage{geometry}               % See geometry.pdf to learn the
                                    % layout options. There are lots.
\usepackage{caption}
\captionsetup{font={stretch=1.5}}
\usepackage{color}
\usepackage{xcolor}
\usepackage{subcaption}  % Emma added it
\usepackage{amsmath,textcomp}
\usepackage{amssymb}
\usepackage{bm,upgreek}
\usepackage{enumerate}
\usepackage[utf8]{inputenc}

%\usepackage{fixmath}
%\usepackage[super,sort&compress,comma]{natbib}
\usepackage[numbers,square,sort&compress,comma]{natbib} 
%\usepackage[version=3]{mhchem}
%\usepackage{times,mathptmx}
%\usepackage{sectsty}
%\usepackage{balance}
%\usepackage{lastpage}
%\usepackage[format=plain,justification=raggedright,singlelinecheck=false,font=small,labelfont=bf,labelsep=space]{caption} 
%\usepackage{fancyhdr}
%\pagestyle{fancy}

%%%Jan Smreck packages
\usepackage{latexsym}
\usepackage{amsfonts}
\usepackage{hhline}
%%%%
\usepackage[separate-uncertainty=true,multi-part-units=single,per=slash,range-phrase=--,range-units=brackets,per-mode=symbol-or-fraction]{siunitx}
%\DeclareSIUnit{u}{\mu}
\usepackage{url}
\usepackage{indentfirst}
\usepackage{suffix}
\usepackage{tabularx}
\usepackage{booktabs}
\usepackage{parskip}
%% Controls spacing between lines (\doublespacing, \onehalfspacing,
%% etc.):
\usepackage{setspace}
\usepackage{hyperref}
\hypersetup{
	colorlinks,
        citecolor=blue,
        urlcolor=black
}
\usepackage[labelfont=bf]{caption}		%boldface caption labels

%\usepackage{xcolor,soul}
%	\sethlcolor{purple}
%\newcommand{\colorcomment}[1]{ {\color{white}\hl{  \bf #1 }}}
%modifying colors for links
%\colorlet{linkequation}{red}
%\usepackage[colorlinks=true,citecolor=blue,linkcolor=purple,urlcolor=black]{hyperref}
%\newcommand*{\SavedEqref}{}
%\let\SavedEqref\eqref
%\renewcommand*{\eqref}[1]{%
%	\begingroup
%	\hypersetup{
%		linkcolor=linkequation,
%		linkbordercolor=linkequation,
%	}%
%	\SavedEqref{#1}%
%	\endgroup
%}




%Define new commands
%%%%%%%%%%%%%%%%%%%%%%%%%%%%%%%%%%%%%%%%%%%%%%%%%%%%%%%%%%%%%%%%%%%%%


\renewcommand{\figurename}{\small{Fig.}~}
\renewcommand{\vec}[1]{\boldsymbol{#1}}
\newcommand{\oldvec}[1]{\vec{#1}}
%\renewcommand{\vec}[1]{\mathbf{#1}}
\newcommand{\uvec}[1]{\boldsymbol{\hat{#1}}}
\newcommand{\abs}[1]{\left\vert #1 \right\vert}
\newcommand{\imaginary}[1]{\Im\left\{ #1 \right\}}
\newcommand{\real}[1]{\Re\left\{ #1 \right\}}

\newcommand{\vecr}{\left(\vec{r}\right)}
\newcommand{\basisj}{\hat{e}_j}
\newcommand{\basis}{\left\{\basisj\right\}}
\newcommand\vectorpotential{\vec{A}\left(\vec{r},t\right)}
\WithSuffix\newcommand\vectorpotential*{\vec{A}^\ast\left(\vec{r},t\right)}
\newcommand{\electricfield}{\vec{E}\left(\vec{r},t\right)}
\WithSuffix\newcommand\electricfield*{\vec{E}^\ast\left(\vec{r},t\right)}
\newcommand{\magneticfield}{\vec{H}(\vec{r},t)}
\WithSuffix\newcommand\magneticfield*{\vec{H}^\ast\left(\vec{r},t\right)}
\newcommand{\amplitude}{u\vecr}
\newcommand{\ph}{\varphi}
\newcommand{\phj}{\ph_j}
\newcommand{\phase}{\ph\vecr}
\newcommand{\phasej}{\phj\vecr}
\newcommand{\pol}{\varepsilon}
\newcommand{\polj}{\pol_j}
\newcommand\polarization{\hat{\pol}\vecr}
\WithSuffix\newcommand\polarization*{\hat{\pol}^\ast\vecr}
\newcommand{\polarizationj}{\polj\vecr}
\WithSuffix\newcommand\polarizationj*{\polj^\ast\vecr}
\newcommand{\intensity}{I\vecr}
\newcommand{\action}{\mathcal{I}\vecr}
\newcommand{\momentum}{\vec{g}\vecr}
\newcommand{\spin}{\vec{s}}
\newcommand{\spindensity}{\spin\vecr}
\newcommand{\helicity}{\pmb{\sigma}\vecr}
\newcommand{\wavevector}{\vec{q}\vecr}
\newcommand{\wavenumber}{q\vecr}
\newcommand{\propagation}{\hat{q}\vecr}
%\newcommand{\orbitaldensity}{\vec{\ell}\vecr}
\newcommand{\orbitaldensity}{\pmb{\ell}\vecr}
\newcommand{\orbital}{\vec{\ell}\vecr}
\newcommand{\Lagr}{\mathcal{L}}

\newcommand{\mytilde}{\raise.17ex\hbox{$\scriptstyle\mathtt{\sim}$}}
\newcommand{\pos}{\left(\mathbf{r} \right)}
\newcommand{\bigN}[1]{\vec{N}_{nm}^{(#1)}(k\vec{r})}
\newcommand{\bigM}[1]{\vec{M}_{nm}^{(#1)}(k\vec{r})}

%Commands used by Jan Smreck for making list of appendices
\newcommand\listappendixname{List of Appendices}
\newcommand\appcaption[1]{%
  \addcontentsline{app}{section}{#1}}
\makeatletter
\newcommand\listofappendices{%
  \chapter*{\listappendixname}\@starttoc{app}}
\makeatother

%Commands hide a chapter from the table of contents
% from the following forum post:
% http://tex.stackexchange.com/questions/20543/excluding-chapters-from-toc-in-ams
\DeclareRobustCommand{\gobblefive}[5]{}
\newcommand*{\SkipTocEntry}{\addtocontents{toc}{\gobblefive}}

\makeatletter % Need for anything that contains an @ command 
\renewcommand{\maketitle} % Redefine maketitle to conserve space
{ \begingroup \vskip 10pt \begin{center} \large {\bf \@title}
	\vskip 10pt \large \@author \hskip 20pt \@date \end{center}
  \vskip 10pt \endgroup \setcounter{footnote}{0} }
\makeatother % End of region containing @ commands
\renewcommand{\labelenumi}{(\alph{enumi})} % Use letters for enumerate
% \DeclareMathOperator{\Sample}{Sample}
\let\vaccent=\v % rename builtin command \v{} to \vaccent{}
\renewcommand{\v}[1]{\ensuremath{\mathbf{#1}}} % for vectors
\newcommand{\gv}[1]{\ensuremath{\mbox{\boldmath$ #1 $}}} 
% for vectors of Greek letters
\newcommand{\uv}[1]{\ensuremath{\mathbf{\hat{#1}}}} % for unit vector
\newcommand{\avg}[1]{\left< #1 \right>} % for average

% For capital chapter references.
\let\orgautoref\autoref
\providecommand{\Autoref}[1]{\def\chapterautorefname{Chapter}\orgautoref{#1}}
\renewcommand{\autoref}[1]{\def\chapterautorefname{chapter}\orgautoref{#1}}
